%%%%%%%%%%%%%%%%%%%%%%%%%%%%%%%%%%%%%%%%%%%%%%%%%%%%%%%%%%%%%%%%%%%%%%%%%%%%%%%
\documentclass[12pt]{report}
\usepackage[utf8]{inputenc}
\usepackage[T1]{fontenc}
%%%%%%%%%%%%%%%%%%%%%%%%%%%%%%%%%%%%%%%%%%%%%%%%%%%%%%%%%%%%%%%%%%%%%%%%%%%%%%%
% GHP derivatives in math mode 
\newcommand{\spindersphere}{{}_s\slashed{\Delta}}
\newcommand{\edth}{\textnormal{\dh}}
\newcommand{\Thorn}{\textnormal{\th}}
\newcommand{\Edth}{\textnormal{\DH}}
\newcommand{\mathTH}{\textnormal{\TH}}
\newcommand{\firstorder}[1]{\dot{#1}}
\newcommand{\secondorder}[1]{\ddot{#1}}
%%%%%%%%%%%%%%%%%%%%%%%%%%%%%%%%%%%%%%%%%%%%%%%%%%%%%%%%%%%%%%%%%%%%%%%%%%%%%%%
% to make overbar (for complex conjugate) easier to read from a distance
\newcommand*\oline[1]{%
   \vbox{%
     \hrule height 1pt%                  % Line above with certain width
     \kern0.25ex%                          % Distance between line and content
     \hbox{%
       \ifmmode#1\else\ensuremath{#1}\fi%  % The content, typeset in dependence of mode
     }
   }
}
%%%%%%%%%%%%%%%%%%%%%%%%%%%%%%%%%%%%%%%%%%%%%%%%%%%%%%%%%%%%%%%%%%%%%%%%%%%%%%%
\usepackage{empheq}
\newcommand*\widefbox[1]{\fbox{\hspace{2em}#1\hspace{2em}}}
\newcommand{\justin}[1]{{\textbf{#1}} }
%%%%%%%%%%%%%%%%%%%%%%%%%%%%%%%%%%%%%%%%%%%%%%%%%%%%%%%%%%%%%%%%%%%%%%%%%%%%%%%
\usepackage{mathalfa,amssymb}
\usepackage{graphicx}
\usepackage{amsmath}
\usepackage{amssymb}
\usepackage{slashed}
\usepackage{graphicx}
\usepackage{setspace}
\usepackage{fullpage}
\usepackage{enumerate}
\usepackage{braket}
\usepackage[hidelinks]{hyperref}

\allowdisplaybreaks

\begin{document}

\title{
{Notes on perturbations of spherically symmetric spacetimes}\\
}
\author{Justin L. Ripley 
   \\ \small{lloydripley[at]gmail[dot]com}
   }
\date{\today}

\maketitle

\abstract{We start by reviewing the Einstein equations in spherical
   symmetry. We then write down the perturbed Einstein equations about a
   spherically symmetric background.
   This is mostly a review of a covariant framework for the spherical
   decomposition of tensors \cite{Martel:2003ab,Martel:2005ir,Gundlach:1999bt,Martin-Garcia:2000cgm}.
   These notes are essentially an outgrowth of notes for the paper
   \cite{Ripley:2017kqg}.
   Please let me know if you find any typos/errors!
}
%==============================================================================
%\allowdisplaybreaks
%\tableofcontents
%==============================================================================
\chapter{General equations of motion}

Our notation generally follows \cite{Wald:1984rg}. For a textbook discussion
of relativistic fluids, see \cite{Rezzolla-Book}.
We consider the Einstein equations coupled to fluid matter 
\begin{align}
    E^{(g)}_{\alpha\beta}
    \equiv
    R_{\alpha\beta}
    -
    \frac{1}{2}g_{\alpha\beta}R
    &=
    \kappa T_{\alpha\beta}
    ,\\
    \nabla_{\alpha}T^{\alpha\beta}
    &=
    0
    ,\\
    E^{(g)}
    \equiv
    \nabla_{\alpha}J^{\alpha}
    &=
    0
    .
\end{align}
Here $T_{\alpha\beta}$ is the stress-energy tensor, and $J^{\alpha}$ is the fluid current. 

We decompose the stress-energy tensor in terms of
the fluid velocity vector $u^{\alpha}$, 
which is a unit timelike vector ($u^{\alpha}u_{\alpha}=-1$):
\begin{align}
    T^{\alpha\beta}
    &=
    \mathcal{E}u^{\alpha}u^{\beta}
    +
    \mathcal{P}\Delta^{\alpha\beta}
    +
    \left(\mathcal{Q}^{\alpha}u^{\beta} + \mathcal{Q}^{\beta}u^{\alpha}\right)
    +
    \mathcal{T}^{\alpha\beta}
    ,\\
    J^{\alpha}
    &=
    \mathcal{N}u^{\alpha}
    +
    \mathcal{J}^{\alpha}
    ,
\end{align}
where $\mathcal{E}$, $\mathcal{P}$, and $\mathcal{N}$ are scalars,
$\mathcal{Q}^{\alpha}$, $\mathcal{J}^{\alpha}$ are vectors transverse to $u^{\alpha}$
(for example $u_{\alpha}\mathcal{Q}^{\alpha}=0$),
and $\mathcal{T}^{\alpha\beta}$ is a symmetric transverse-traceless tensor
with respect to $u^{\alpha}$ (that is $u_{\alpha}\mathcal{T}^{\alpha\beta}=\mathcal{T}^{\alpha}{}_{\alpha}=0$,
and
\begin{align}
   \Delta^{\alpha\beta}
   \equiv
   g^{\alpha\beta}
   +
   u^{\alpha}u^{\beta}
   ,
\end{align}
projects onto the space transverse to $u^{\alpha}$.
More specifically, for a $d$ dimensional spacetime
(we work in $d=4$ spacetime dimensions) we have
\begin{subequations}
\begin{align}
   \mathcal{E}
   &\equiv
    u_{\alpha}u_{\beta}T^{\alpha\beta}
   ,\\
   \mathcal{P}
   &\equiv 
   \frac{1}{d-1}\Delta_{\alpha\beta}T^{\alpha\beta}
   ,\\
   \mathcal{Q}_{\alpha}
   &\equiv
   -
    \Delta_{\alpha\beta}u_{\gamma}T^{\beta\gamma}
   ,\\
   \mathcal{N}
   &\equiv
   -
    u_{\gamma}J^{\gamma}
   ,\\
   \mathcal{J}_{\alpha}
   &\equiv
   \Delta_{\alpha\beta}J^{\beta}
   ,\\
   \mathcal{T}^{\alpha\beta}
   &\equiv
   T^{\left<\alpha\beta\right>}
   ,
\end{align}
\end{subequations}
where the angle brackets of a tensor is defined to be
the symmetric transverse-traceless part of the tensor
\begin{align}
   X^{\left<\alpha\beta\right>}
   &\equiv
   \frac{1}{2}\left(
      \Delta^{\alpha\gamma}\Delta^{\beta\delta}
      \left(X_{\gamma\delta} + X_{\delta\gamma}\right)
      -
      \frac{2}{d-1}\Delta^{\alpha\beta}
      \Delta^{\gamma\delta}X_{\gamma\delta}
   \right)
   .
\end{align}
So far we have only given a general decomposition of the stress-energy
tensor with respect to a timelike unit vector $u^{\alpha}$.
Specifying a specific fluid theory requires specifying 
\emph{constitutive relations} for the quantities 
$\mathcal{E}$, ..., $\mathcal{T}^{\alpha\beta}$. 
It's worth noting that the trace of the stress-energy tensor is
\begin{align}
    T
    =
    -
    \mathcal{E}
    +
    3\mathcal{P}
    ,
\end{align}
that is, the heat flux and shear do not contribute to the trace.
The conservation equation $\nabla_{\alpha}T^{\alpha\beta}=0$ can be split into 
a part parallel to $u^{\beta}$ and perpendicular to $u^{\beta}$ 
(the relativistic generalizations of the continuity and
Euler-Navier-Stokes equations):
\begin{align}
    u^{\alpha}\nabla_{\alpha}\mathcal{E}
    +
    \left(\mathcal{E} + \mathcal{P}\right)\nabla_{\alpha}u^{\alpha}
    +
    \nabla_{\alpha}\mathcal{Q}^{\alpha}
    -
    u_{\gamma}u^{\alpha}\nabla_{\alpha}\mathcal{Q}^{\gamma}
    -
    u_{\gamma}\nabla_{\alpha}\mathcal{T}^{\alpha\gamma}
    &=
    0
    ,\\
    \left(\mathcal{E} + \mathcal{P}\right)u^{\alpha}\nabla_{\alpha}u^{\beta}
    +
    \mathcal{Q}^{\alpha}\nabla_{\alpha}u^{\beta}
    +
    \mathcal{Q}^{\beta}\nabla_{\alpha}u^{\alpha}
    +
    \Delta^{\alpha\beta}\nabla_{\alpha}\mathcal{P}
    \qquad
    &\nonumber\\
    +
    \Delta^{\beta}{}_{\gamma}u^{\alpha}\nabla_{\alpha}\mathcal{Q}^{\gamma}
    +
    \Delta^{\beta}{}_{\gamma}\nabla_{\alpha}\mathcal{T}^{\alpha\gamma}
    &=
    0
    .
\end{align}
The conservation of the current, $\nabla_{\alpha}J^{\alpha}=0$, can be written as
\begin{align}
   u^{\alpha}\nabla_{\alpha}\mathcal{N}
   +
   \mathcal{N}\nabla_{\alpha}u^{\alpha}
   +
   \nabla_{\alpha}\mathcal{J}^{\alpha}
   =
   0
   .
\end{align}

We consider perturbations of non-rotating neutron star solutions, that
is perturbations of the Einstein-fluid system:
\begin{align}
   \delta \left(
        R_{\alpha\beta} 
        - 
        \kappa 
        \left(
            T_{\alpha\beta}
            -
            \frac{1}{2}g_{\alpha\beta}T
        \right)
    \right) 
   &= 
   0 
   ,\\
   \delta \left(\nabla_{\alpha}T^{\alpha\beta}\right)
   &=
   0
   ,\\
   \delta\left(\nabla_{\alpha}J^{\alpha}\right)
   &=
   0
   .
\end{align}

As we are perturbing about a spherically symmetric background,
we can decompose linear perturbations according to how they transform
under rotations (irreducible components of the rotation group).
We consider perturbations of the metric $\delta g_{\alpha\beta}$, 
and Eulerian perturbations $\delta u^{\alpha}$ of the fluid velocity.

%===========================================================================
\section{Perturbation of the Ricci tensor\label{eq:pert_ricci_tensor}}
We start with the well-known identities \cite{Wald:1984rg} 
\begin{align}
    \label{eq:general_pert_riemann_tensor}
    \delta R^{\alpha}{}_{\gamma\beta\delta}
    &=
    \nabla_{\beta}\delta\Gamma^{\alpha}_{\delta\gamma}
    -
    \nabla_{\delta}\delta\Gamma^{\alpha}_{\beta\gamma}
    ,\\
    \label{eq:general_pert_christoffel_symbol}
    \delta \Gamma^{\gamma}_{\alpha\beta}
    &=
    \frac{1}{2}g^{\gamma\delta}
    \left(
        \nabla_{\alpha}\delta g_{\delta\beta}
        +
        \nabla_{\beta}\delta g_{\delta\alpha}
        -
        \nabla_{\delta}\delta g_{\alpha\beta}
    \right)
    .
\end{align}
We then have
\begin{align}
    \delta R_{\alpha\beta}
    =&
    \nabla_{\gamma}\delta\Gamma^{\gamma}_{\alpha\beta}
    -
    \nabla_{\beta}\delta\Gamma^{\gamma}_{\alpha\gamma}
    \nonumber\\
    =&
    -
    \frac{1}{2}g^{\gamma\delta}
    \nabla_{\gamma}\nabla_{\delta}\delta g_{\alpha\beta}
    +
    \frac{1}{2}g^{\gamma\delta}
    \left(
        \nabla_{\alpha}\nabla_{\gamma}\delta g_{\delta\beta}
        +
        \nabla_{\beta}\nabla_{\delta}\delta g_{\alpha\gamma}
        -
        \nabla_{\beta}\nabla_{\alpha}\delta g_{\delta\gamma}
    \right)
    \nonumber\\
    &
    +
    \frac{1}{2}g^{\gamma\delta}
    \left[\nabla_{\gamma},\nabla_{\beta}\right]\delta g_{\delta\alpha}
    +
    \frac{1}{2}g^{\gamma\delta}
    \left[\nabla_{\gamma},\nabla_{\alpha}\right]\delta g_{\delta\beta}
    \nonumber\\
    =&
    -
    \frac{1}{2}g^{\gamma\delta}
    \nabla_{\gamma}\nabla_{\delta}\delta g_{\alpha\beta}
    +
    \nabla_{(\alpha}v_{\beta)}
    -
    R_{\alpha}{}^{\gamma}{}_{\beta}{}^{\delta}\delta g_{\gamma\delta}
    +
    R^{\gamma}{}_{(\alpha}\delta g_{\beta)\gamma}
    ,
\end{align}
where we have defined
\begin{align}
    \label{eq:linear_v_vector}
    \boxed{
        v_{\mu}
        \equiv
        g^{\gamma\delta}
        \left(
            \nabla_{\gamma}\delta g_{\delta \mu}
            -
            \frac{1}{2}\nabla_{\mu}\delta g_{\gamma\delta}
        \right)
        =
        g^{\gamma\delta} \delta \Gamma_{\mu\gamma\delta}
        .
    }
\end{align}
Putting everything together, we have
\begin{align}
    \label{eq:general_pert_Ricci_tensor}
    \boxed{
        \delta R_{\alpha\beta}
        =
        -
        \frac{1}{2}g^{\gamma\delta}
        \nabla_{\gamma}\nabla_{\delta}\delta g_{\alpha\beta}
        -
        R_{\alpha}{}^{\gamma}{}_{\beta}{}^{\delta}\delta g_{\gamma\delta}
        +
        R^{\gamma}{}_{(\alpha}\delta g_{\beta)\gamma}
        +
        \nabla_{(\alpha}v_{\beta)}
    }
\end{align}
Some authors define the Lichnerowicz wave operator
\begin{align}
    \label{eq:Lich_wav_op}
    \Box_L \delta g_{\alpha\beta}
    \equiv
    -
    \frac{1}{2}g^{\gamma\delta}
    \nabla_{\gamma}\nabla_{\delta}\delta g_{\alpha\beta}
    -
    R_{\alpha}{}^{\gamma}{}_{\beta}{}^{\delta}\delta g_{\gamma\delta}
    +
    R^{\gamma}{}_{(\alpha}\delta g_{\beta)\gamma}
    .
\end{align}
The term $\nabla_{(\alpha}v_{\beta)}$ can be thought as 
describing pure gauge fluctuations in the linearized Einstein equations.
We see the linearized Einstein equations essentially take the
form of a system of linear wave equations for the components
of $\delta g_{\alpha\beta}$.
%===========================================================================
\chapter{Spherically symmetric spacetime decomposition}

\section{Spherically symmetric spacetime}
The metric for a spherically symmetric spacetime can generally be split
into the form (for a review see \cite{Abreu:2010ru})
\begin{align}
    \label{eq:general_spherical_sym_metric}
    \boxed{
        ds^2
        =
        \alpha_{ab}dx^adx^b
        +
        r^2\Omega_{AB}d\theta^Ad\theta^B
        ,
    }
\end{align}
where $r$ (the areal radius) depends on the coordinates $x^a$
(e.g. $x^a=(t,r)$ in Schwarzschild-like coordinates).
Here $\Omega_{AB}$ is the metric for the unit two-sphere.
We define the metric compatible derivative for $\alpha_{ab}$ with $D_a$,
and the metric compatible derivative for $\Omega_{AB}$ with $D_A$.
The Ricci scalar for $\alpha_{ab}$ is $\mathcal{R}$, and the
Ricci scalar for $\Omega_{AB}$ is $2$.
We raise/lower lower case Latin indices with $\alpha_{ab}/\alpha^{ab}$, 
and raise/lower upper case Latin indices with $\Omega_{AB}/\Omega^{AB}$.
We define $r_a\equiv D_ar$, $r_{ab}\equiv D_aD_br$, and so on.
We denote the Lie derivative with respect to a vector $\xi^{\mu}$ with
$\mathcal{L}_{\xi}$.

The nonzero Christoffel symbol components are
\begin{subequations}
\label{eq:chr_sym_spherical_sym}
\begin{align}
    \Gamma^c_{ab}
    &=
    {}^{(2)}\Gamma^c_{ab}
    ,\\
    \Gamma^c_{AB}
    &=
    -
    \Omega_{AB} r r^c
    ,\\
    \Gamma^C_{aB}
    &=
    \delta^C_B\frac{1}{r}r_a
    ,\\
    \Gamma^C_{AB}
    &=
    {}^{(2)}\Gamma^C_{AB}
    .
\end{align}
\end{subequations}

The nonzero components of the Riemann and Ricci tensors, along
with the Ricci scalar, are 
\begin{subequations}
\label{eq:riemann_ricci_spherical_sym}
\begin{align}
    R_{abcd}
    &=
    \frac{1}{2}
    \mathcal{R}
    \left(
        \alpha_{ac}\alpha_{bd}
        -
        \alpha_{ad}\alpha_{bc}
    \right)
    ,\\
    R_{aAbB}
    &=
    -
    r
    r_{ab}
    \Omega_{AB}
    ,\\
    R_{ABCD}
    &=
    \left(
        1
        -
        r_ar^a
    \right)
    r^2
    \left(
        \Omega_{AC}\Omega_{BD}
        -
        \Omega_{AD}\Omega_{BC}
    \right)
    ,\\
    R_{ab}
    &=
    \frac{1}{2}\mathcal{R}\alpha_{ab}
    -
    \frac{2}{r}r_{ab}
    ,\\
    R_{AB}
    &=
    \left(
        1
        -
        r_ar^a 
        -
        r r_a^a
    \right)
    \Omega_{AB}
    ,\\
    R
    &=
    \mathcal{R}
    -
    \frac{4}{r}r_c^c
    +
    \frac{2}{r^2}\left(1-r_ar^a\right)
    .
\end{align}
\end{subequations}

The covariant Misner-Sharp mass $m$ is defined by
\begin{align}
\label{eq:covariant_ms}
    \boxed{
        1
        -
        \frac{2m}{r}
        \equiv
        D_arD^ar
        .
    }
\end{align}

The spherically symmetric stress-energy tensor can be written as
\begin{align}
    T_{\alpha\beta}dx^{\alpha}dx^{\beta}
    =
    T_{ab}dx^adx^b
    +
    T_2 r^2\Omega_{AB}d\theta^Ad\theta^B
    .
\end{align}
Making use of the fluid-decomposition of the stress-energy tensor,
we can also write this as
\begin{empheq}[box=\widefbox]{align}
    T_{\alpha\beta}dx^{\alpha}dx^{\beta}
    =&
    \left(
        \mathcal{E}u_au_b
        +
        \left(
            \mathcal{P}
            +
            \mathcal{T}
        \right)
        \Delta_{ab}
        +
        \mathcal{Q}_au_b
        +
        \mathcal{Q}_bu_a
    \right)
    dx^adx^b
    \nonumber\\
    &
    +
    \left(
        \mathcal{P}
        -
        \frac{1}{2}\mathcal{T}
    \right)
    r^2\Omega_{AB}d\theta^Ad\theta^B
    ,
\end{empheq}
where $\Delta_{ab}\equiv u_au_b+\alpha_{ab}$.
Generally we can include a shear term in
spherical symmetry, although it is not present in perfect fluids.
With this, the spherical decomposition of the 
spherically symmetric Einstein equations are
\begin{align}
    \frac{2}{r}\left(
        \alpha_{ab}r_c^c
        -
        r_{ab} 
    \right)
    -
    \frac{1}{r^2}\left(1 - r_cr^c\right)\alpha_{ab}
    &=
    \kappa\left(
        \left(
            \mathcal{E} 
            + 
            \mathcal{P}
            +
            \mathcal{T}
        \right)
        u_au_b
        +
        \mathcal{P}\alpha_{ab}
        +
        \mathcal{Q}_au_b
        +
        \mathcal{Q}_au_b
    \right)
    \\
    \frac{1}{r}r_c^c
    -
    \frac{1}{2}\mathcal{R}
    &=
    \kappa 
    \left(
        \mathcal{P}
        -
        \frac{1}{2}\mathcal{T}
    \right)
    .
\end{align}

The spherical decomposition of the fluid equations are
\begin{align}
    u^aD_a\mathcal{E}
    +
    \left(\mathcal{E} 
    + 
    \mathcal{P}\right)\frac{1}{r^2}D_a\left(r^2u^a\right)
    +
    \frac{1}{r^2}D_a\left(r^2\mathcal{Q}^a\right)
    +
    u_bu^aD_a\mathcal{Q}^b
    &=
    0
    ,\\
    u^aD_au^b
    +
    Q^aD_au^b
    +
    Q^b\frac{1}{r^2}D_a\left(r^2u^a\right)
    +
    \Delta^{ab}D_a\mathcal{P}
    +
    \Delta^b_cu^aD_a\mathcal{Q}^c
    &=
    0
    .
\end{align}

The spherical decomposition of the conservation of the current is
\begin{align}
    u^aD_a\mathcal{N}
    +
    \mathcal{N}\frac{1}{r^2}D_a\left(r^2 u^a\right)
    +
    \frac{1}{r^2}D_a\left(r^2\mathcal{J}^a\right)
    =
    0
    .
\end{align}

%===========================================================================
\section{Perturbation of a spherically symmetric spacetime
\label{sec:pert_spherically_symmetric_spacetime}}

Following \cite{Martel:2005ir}, we consider linear perturbations of
a spherically symmetric metric
\begin{align}
    ds^2 
    =
    \left(\alpha_{ab} + p_{ab}\right)dx^adx^b
    +
    2 p_{aA}dx^ad\theta^A
    +
    \left(
        r^2
        \Omega_{AB}
        +
        p_{AB}
    \right)
    d\theta^Ad\theta^B
    .
\end{align}
The inverse metric to linear order is
\begin{subequations}
\begin{align}
    g^{ab}
    =&
    \alpha^{ab}
    -
    p^{ab}
    ,\\
    g^{aB}
    =&
    -
    \frac{1}{r^2}p^{aB}
    ,\\
    g^{AB}
    =&
    \frac{1}{r^2}\Omega^{AB}
    -
    \frac{1}{r^4}p^{AB}
    .
\end{align}
\end{subequations}
The perturbations decomposed with respect to irreducible representations
of the rotation group are
\begin{subequations}
\begin{align}
    p_{ab}
    =&
    \sum_{\ell,m} \left[h^m_{\ell}\right]_{ab} Y^m_{\ell}
    ,\\
    p_{aA}
    =&
    \sum_{\ell,m} \left(
        \left[j^m_{\ell}\right]_a \left[E^m_{\ell}\right]_A 
        +
        \left[h^m_{\ell}\right]_a \left[S^m_{\ell}\right]_A
    \right)
    ,\\
    p_{AB}
    =&
    \sum_{\ell,m} \left(
        r^2\left[k^m_{\ell}\right]\Omega_{AB} Y^m_{\ell}
        +
        r^2\left[g^m_{\ell}\right]\left[Z^m_{\ell}\right]_{AB}
        +
        \left[h^m_{\ell}\right]_2 \left[S^m_{\ell}\right]_{AB}
    \right)
    .
\end{align}
\end{subequations}

We next review how to construct gauge-invariant linear perturbations
in the spherical harmonic decomposition \cite{Martel:2005ir}.
Consider linear gauge transformations
\begin{align}
    g^{\prime}_{\alpha\beta}
    &= 
    g_{\alpha\beta}
    -
    \mathcal{L}_{\Xi}g_{\alpha\beta}
    \nonumber\\
    &= 
    g_{\alpha\beta}
    -
    \nabla_{\alpha}\Xi_{\beta}
    -
    \nabla_{\beta}\Xi_{\alpha}
    .
\end{align}
We decompose the gauge transformation vector as
\begin{subequations}
\begin{align}
    \Xi_a
    &=
    \sum_{\ell,m} \left[\xi^m_{\ell}\right]_a Y^m_{\ell}
    \\
    \Xi_A
    &=
    \sum_{\ell,m}\left(
        \left[\xi^m_{\ell}\right]_+ \left[E^m_{\ell}\right]_A
        +
        \left[\xi^m_{\ell}\right]_- \left[S^m_{\ell}\right]_A
    \right)
    .
\end{align}
\end{subequations}

The components of the linearized metric transform as 
\begin{subequations}
\begin{align}
    \left[h^m_{\ell}\right]^{\prime}_{ab}
    =&
    \left[h^m_{\ell}\right]_{ab}
    -
    D_a\left[\xi^m_{\ell}\right]_b
    -
    D_b\left[\xi^m_{\ell}\right]_a
    \\
    \left[j^m_{\ell}\right]^{\prime}_a
    =&
    \left[j^m_{\ell}\right]_a
    -
    \left[\xi^m_{\ell}\right]_a
    -
    D_a\left[\xi^m_{\ell}\right]_+
    +
    \frac{2}{r}r_a\left[\xi^m_{\ell}\right]_+
    ,\\
    \left[k^m_{\ell}\right]^{\prime}
    =&
    \left[k^m_{\ell}\right]
    +
    \frac{\ell\left(\ell+1\right)}{r^2}\left[\xi^m_{\ell}\right]_+
    -
    \frac{2}{r}r^a\left[\xi^m_{\ell}\right]_a
    ,\\
    \left[g^m_{\ell}\right]^{\prime}
    =&
    \left[g^m_{\ell}\right]
    -
    \frac{2}{r^2}\left[\xi^m_{\ell}\right]_+
    ,\\
    \left[h^m_{\ell}\right]_a^{\prime}
    =&
    \left[h^m_{\ell}\right]_a
    -
    D_a\left[\xi^m_{\ell}\right]_-
    +
    \frac{2}{r}r_a\left[\xi^m_{\ell}\right]_-
    ,\\
    \left[h^m_{\ell}\right]^{\prime}_2
    =&
    \left[h^m_{\ell}\right]_2
    -
    2\left[\xi^m_{\ell}\right]_-
    .
\end{align}
\end{subequations}
Gauge-invariant combinations of these variables are
\begin{align}
    \left[\tilde{h}^m_{\ell}\right]_{ab}
    \equiv&
    \left[h^m_{\ell}\right]_{ab}
    -
    D_a\left[\varepsilon^m_{\ell}\right]_b
    -
    D_b\left[\varepsilon^m_{\ell}\right]_a
    ,\\
    \left[\tilde{k}^m_{\ell}\right]
    \equiv&
    \left[k^m_{\ell}\right]
    +
    \frac{1}{2}\ell\left(\ell+1\right)\left[g^m_{\ell}\right]
    -
    \frac{2}{r}r^a\left[\varepsilon^m_{\ell}\right]_a
    ,\\
    \left[\tilde{h}^m_{\ell}\right]_a
    \equiv&
    \left[h^m_{\ell}\right]_a
    -
    \frac{1}{2}D_a\left[h^m_{\ell}\right]_2
    +
    \frac{1}{r}r_a\left[h^m_{\ell}\right]_2
    .
\end{align}
where
\begin{align}
    \left[\varepsilon^m_{\ell}\right]_a
    \equiv
    \left[j^m_{\ell}\right]_a
    -
    \frac{1}{2}r^2\left[g^m_{\ell}\right]
    .
\end{align}
In the Regge-Wheeler gauge \cite{Regge:1957td,1967ApJ...149..591T}
\begin{align}
\label{eq:regge_wheeler_gauge}
    \boxed{
        \left[j^m_{\ell}\right]_a
        =
        \left[g^m_{\ell}\right]
        =
        \left[h^m_{\ell}\right]_2
        =
        0
        .
    }
\end{align}
In this gauge we see that 
\begin{align}
\label{eq_regge_wheeler_gauge_invar}
    \boxed{
        \left[\tilde{h}^m_{\ell}\right]_{ab}
        = 
        \left[h^m_{\ell}\right]_{ab}
        ,\qquad
        \left[\tilde{k}^m_{\ell}\right]
        =
        \left[k^m_{\ell}\right]
        ,\qquad
        \left[\tilde{h}^m_{\ell}\right]_a
        =
        \left[h^m_{\ell}\right]_a
        .
    }
\end{align}
We can then derive the polar and axial equations of motion in Regge-Wheeler
gauge, and then make those equations gauge invariant
by promoting 
$\left[h^m_{\ell}\right]_{ab}\to\left[\tilde{h}^m_{\ell}\right]_{ab}$,
$\left[k^m_{\ell}\right]\to\left[\tilde{k}^m_{\ell}\right]$,
and
$\left[h^m_{\ell}\right]_a\to\left[\tilde{h}^m_{\ell}\right]_a$ 
\cite{Martel:2005ir}.

Similarly to how we decompose the perturbations of the spacetime metric, 
we can decompose the perturbations of the stress-energy tensor
\begin{align}
    T^{\alpha}_{\beta}dx^{\beta}\partial_{\alpha}
    =
    \left(
        T^{a}_{b} + P^{a}_{b}
    \right)
    dx^b\partial_a
    +
    P^{a}_{B}d\theta^B\partial_a
    +
    P^{A}_{b}d\theta^b\partial_A
    +
    \left(
        \delta^A_B\mathcal{P}
        +
        P^{A}_{B}
    \right)
    d\theta^B\partial_A
    ,
\end{align}
and set
\begin{subequations}
\begin{align}
    P^a_b
    =&
    \sum_{\ell,m} \left[H^m_{\ell}\right]^a_{b} Y^m_{\ell}
    ,\\
    P^a_B
    =&
    \sum_{\ell,m} \left(
        \left[J^m_{\ell}\right]^a \left[E^m_{\ell}\right]_B 
        +
        \left[H^m_{\ell}\right]^a \left[S^m_{\ell}\right]_B
    \right)
    ,\\
    P^A_b
    =&
    \frac{1}{r^2}\Omega^{AC}\alpha_{bc}P^{c}_{C}
    ,\\
    P^A_{B}
    =&
    \sum_{\ell,m} \left(
        \left[K^m_{\ell}\right]\delta^A_B Y^m_{\ell}
        +
        \left[G^m_{\ell}\right]\left[Z^m_{\ell}\right]^A_{B}
        +
        \frac{1}{r^2}\left[H^m_{\ell}\right]_2 \left[S^m_{\ell}\right]^A_{B}
    \right)
    .
\end{align}
\end{subequations}

Under linear gauge transformations, we have
\begin{align}
    \left(T^{\prime}\right)^{\alpha}_{\beta}
    =&
    T^{\alpha}_{\beta}
    -
    \mathcal{L}_{\Xi}T^{\alpha}_{\beta}
    \nonumber\\
    =&
    T^{\alpha}_{\beta}
    -
    \Xi^{\gamma}\nabla_{\gamma}T^{\alpha}_{\beta}
    +
    T^{\gamma}_{\beta}\nabla_{\gamma}\Xi^{\alpha}
    -
    T^{\alpha}_{\gamma}\nabla_{\beta}\Xi^{\gamma}
    .
\end{align}
If $T_{\alpha\beta}=0$ on the background, then the linearized
stress-energy tensor perturbations are gauge invariant. 
That is not generally the case for us, though.
In general we have
\begin{subequations}
\begin{align}
    \left(\left[H^m_{\ell}\right]^{\prime}\right)^a_{b}
    =&
    \left[H^m_{\ell}\right]^a_{b}
    -
    \left[\xi^m_{\ell}\right]^cD_cT^a_{b}
    +
    T^{c}_{b}D_c\left[\xi^m_{\ell}\right]^a
    -
    T^{a}_{c}D_b\left[\xi^m_{\ell}\right]^c
    \\
    \left(\left[J^m_{\ell}\right]^{\prime}\right)^a
    =&
    \left[J^m_{\ell}\right]^a
    +
    \mathcal{P}\left[\xi^m_{\ell}\right]^a 
    -
    T^a_c\left[\xi^m_{\ell}\right]^c 
    ,\\
    \left[K^m_{\ell}\right]^{\prime}
    =&
    \left[K^m_{\ell}\right]
    -
    \left[\xi^m_{\ell}\right]^cD_c\mathcal{P}
    ,\\
    \left[G^m_{\ell}\right]^{\prime}
    =&
    \left[G^m_{\ell}\right]
    ,\\
    \left(\left[H^m_{\ell}\right]^{\prime}\right)^a
    =&
    \left[H^m_{\ell}\right]^a
    ,\\
    \left[H^m_{\ell}\right]^{\prime}_2
    =&
    \left[H^m_{\ell}\right]_2
    .
\end{align}
\end{subequations}
%-----------------------------------------------------------------------------
\subsection{Perturbation of the Christoffel symbols}
To compute the linearized equations of motion, we need
the perturbed Christoffel symbol components.
Using standard formulas \cite{Wald:1984rg}, we have
\begin{align}
    \delta \Gamma^{\gamma}_{\alpha\beta}
    =&
    \frac{1}{2}g^{\gamma\delta}\left(
        \nabla_{\alpha}\delta g_{\delta\beta}
        +
        \nabla_{\beta}\delta g_{\delta\alpha}
        -
        \nabla_{\delta}\delta g_{\alpha\beta}
    \right)
    \nonumber\\
    =&
    \frac{1}{2}g^{\gamma\delta}\left(
        \partial_{\alpha}\delta g_{\delta\beta}
        +
        \partial_{\beta}\delta g_{\delta\alpha}
        -
        \partial_{\delta}\delta g_{\alpha\beta}
    \right)
    -
    g^{\gamma\delta}\Gamma^{\rho}_{\alpha\beta}
    \delta g_{\delta\rho}
    .
\end{align}
We then have
\begin{align}
    \delta\Gamma^c_{ab}
    =&
    \frac{1}{2}\alpha^{cd}
    \left(
        \partial_ap_{db}
        +
        \partial_bp_{da}
        -
        \partial_dp_{ab}
    \right)
    -
    \alpha^{cd}\Gamma^{\rho}_{ab}p_{d\rho}
    \nonumber\\
    =&
    C^c_{ab}
    ,\\
    \delta\Gamma^C_{ab}
    =&
    \frac{1}{2}\frac{1}{r^2}\Omega^{CD}
    \left(
        \partial_ap_{Db}
        +
        \partial_bp_{Da}
        -
        \partial_Dp_{ab}
    \right)
    -
    \frac{1}{r^2}\Omega^{CD}\Gamma^{\rho}_{ab}p_{D\rho}
    \nonumber\\
    =&
    \frac{1}{2}\frac{1}{r^2}
    \left(
        D_ap^C_b
        +
        D_bp^C_a
        -
        D^Cp_{ab}
    \right)
    ,\\
    \delta\Gamma^c_{Ab}
    =&
    \frac{1}{2}\alpha^{cd}
    \left(
        \partial_Ap_{db}
        +
        \partial_bp_{dA}
        -
        \partial_dp_{Ab}
    \right)
    -
    \alpha^{cd}\Gamma^{\rho}_{Ab}p_{d\rho}
    \nonumber \\
    =&
    \frac{1}{2}
    \left(
        D_Ap^c_b
        +
        D_bp^c_A
        -
        D^cp_{bA}
    \right)
    -
    \left(\frac{1}{r}D_br\right) p^c_A
    ,\\
    \delta\Gamma^c_{AB}
    =&
    \frac{1}{2}\alpha^{cd}
    \left(
        \partial_Ap_{dB}
        +
        \partial_Bp_{dA}
        -
        \partial_dp_{AB}
    \right)
    -
    \alpha^{cd}\Gamma^{\rho}_{AB}p_{d\rho}
    \nonumber\\
    =&
    \frac{1}{2}
    \left(
        D_Ap^c_B
        +
        D_Bp^c_A
        -
        D^cp_{AB}
    \right)
    +
    \Omega_{AB}\left(rD_dr\right)p^{cd}
    ,\\
    \delta\Gamma^C_{Ab}
    =&
    \frac{1}{2}\frac{1}{r^2}\Omega^{CD}
    \left(
        \partial_Ap_{Db}
        +
        \partial_bp_{DA}
        -
        \partial_Dp_{Ab}
    \right)
    -
    \frac{1}{r^2}\Omega^{CD}\Gamma^{\rho}_{Ab}p_{D\rho}
    \nonumber\\
    =&
    \frac{1}{2}\frac{1}{r^2}
    \left(
        D_Ap^C_b
        +
        D_bp^C_A
        -
        D^Cp_{Ab}
    \right)
    -
    \left(\frac{1}{r^3}D_br\right)p^C_A
    ,\\
    \delta\Gamma^C_{AB}
    =&
    \frac{1}{r^2}\Omega^{CD}
    \left(
        \partial_Ap_{DB}
        +
        \partial_Bp_{DA}
        -
        \partial_Dp_{AB}
    \right)
    -
    \frac{1}{r^2}\Omega^{CD}\Gamma^{\rho}_{AB}p_{D\rho}
    \nonumber\\
    =&
    \frac{1}{r^2}C^C_{AB}
    +
    \left(\frac{1}{r}D_pr\right)\Omega_{AB}p^{Cp}
    ,
\end{align}
where we have defined
\begin{align}
    C^c_{ab}
    \equiv&
    \frac{1}{2}\alpha^{cd}
    \left(
        D_ap_{db}
        +
        D_bp_{da}
        -
        D_dp_{ab}
    \right)
    \nonumber\\
    C^C_{AB}
    \equiv&
    \frac{1}{2}\Omega^{CD}
    \left(
        D_Ap_{DB}
        +
        D_Bp_{DA}
        -
        D_Dp_{AB}
    \right)
    .
\end{align}
Notice that we raise/lowered the capital Latin indices with
the metric $\Omega^{AB}$/$\Omega_{AB}$, without any factors of $r$.

%===========================================================================
\section{Perturbation of the stress-energy tensor and conserved vector 
about spherical symmetry\label{sec:pert_stress_energy_tensor_ss}}

We consider linear perturbations about a spherically symmetric metric
of the fluid stress-energy tensor.
The general equation is
\begin{align}
    \delta \left(\nabla_{\gamma}T^{\gamma}_{\alpha}\right)
    =&
    \delta \left(
        \partial_{\gamma}T^{\gamma}_{\alpha}
        +
        \Gamma^{\gamma}_{\gamma\beta}T^{\beta}_{\alpha}
        -
        \Gamma^{\beta}_{\gamma\alpha}T^{\gamma}_{\beta}
    \right)
    \nonumber\\
    =&
    \partial_{\gamma}\delta T^{\gamma}_{\alpha}
    +
    \delta \Gamma^{\gamma}_{\gamma\beta}T^{\beta}_{\alpha}
    +
    \Gamma^{\gamma}_{\gamma\beta}\delta T^{\beta}_{\alpha}
    -
    \delta \Gamma^{\beta}_{\gamma\alpha}T^{\gamma}_{\beta}
    -
    \Gamma^{\beta}_{\gamma\alpha}\delta T^{\gamma}_{\beta}
    .
\end{align}
The different components are
\begin{subequations}
\begin{align}
    \label{eq:linear_pert_se_eom_a_comp}
    \delta\left(\nabla_{\gamma}T^{\gamma}_{a}\right)
    =&
    \partial_{c}P^{c}_{a}
    +
    \partial_{C}P^{C}_{a}
    +
    \left(
        \Gamma^{c}_{cb}
        +
        \Gamma^{C}_{Cb}
    \right)
    P^{b}_{a}
    +
    \Gamma^{C}_{CB}P^{B}_{a}
    -
    \Gamma^b_{ca}P^c_b
    -
    \Gamma^B_{Ca}P^C_B
    \nonumber\\
    &
    +
    \left(
        \delta\Gamma^c_{cb}
        +
        \delta\Gamma^C_{Cb}
    \right)
    T^b_a
    -
    \delta\Gamma^b_{ca}T^c_b
    -
    \delta\Gamma^B_{Ca}T^C_B
    \nonumber\\
    =&
    D_cP^{c}_{a}
    +
    D_CP^{C}_{a}
    +
    \frac{2}{r}r_cP^{c}_{a}
    -
    \frac{1}{r}r_aP^C_C
    \nonumber\\
    &
    -
    C^b_{ca}T^c_b
    +
    \left(
        C^c_{cb}
        +
        \frac{1}{2r^2}D_bp^C_C
        -
        \frac{1}{r^3}r_bp^C_C
    \right)
    T^b_a
    -
    \left(
        \frac{1}{2r^2}D_ap^C_C
        -
        \frac{1}{r^3}r_ap^C_C
    \right)
    \mathcal{P}
    \\
    %%%%%%%%%%%%%%%%%%%%%%%%%%%%%
    \label{eq:linear_pert_se_eom_A_comp}
    \delta\left(\nabla_{\gamma}T^{\gamma}_{A}\right)
    =&
    \partial_cP^{c}_{A}
    +
    \partial_CP^{C}_{A}
    +
    \left(
        \Gamma^{c}_{cb}
        +
        \Gamma^C_{Cb}
    \right)
    P^{b}_{A}
    +
    \Gamma^{C}_{CB}P^{B}_{A}
    -
    \Gamma^B_{CA}P^C_B
    \nonumber\\
    &
    +
    \left(  
        \delta\Gamma^c_{cB}
        +
        \delta\Gamma^C_{CB}
    \right)
    T^B_A
    -
    \delta\Gamma^b_{cA}T^c_b
    -
    \delta\Gamma^B_{CA}T^C_B
    \nonumber\\
    =&
    D_cP^{c}_{A}
    +
    D_CP^{C}_{A}
    +
    \frac{2}{r}r_cP^{c}_{A}
    \nonumber\\
    &
    +
    \left(
        \frac{1}{2}D_Ap^c_c
        -
        \frac{1}{r}r_cp^c_A
    \right)
    \mathcal{P}
    -
    \left(
        \frac{1}{2}
        D_Ap^b_c
        -
        \frac{1}{r}r_cp^b_A
    \right)
    T^c_b
    .
\end{align}
\end{subequations}
%============ ===============================================================
\section{Perturbation of the Ricci tensor in spherical symmetry}

We will use the formula 
\begin{align}
    \delta R_{\alpha\beta}
    =
    -
    \frac{1}{2}g^{\gamma\delta}
    \nabla_{\gamma}\nabla_{\delta}\delta g_{\alpha\beta}
    -
    R_{\alpha}{}^{\gamma}{}_{\beta}{}^{\delta}\delta g_{\gamma\delta}
    +
    R^{\gamma}{}_{(\alpha}\delta g_{\beta)\gamma}
    +
    \nabla_{(\alpha}v_{\beta)} 
\end{align}
where,
\begin{align}
    v_{\alpha} \equiv g^{\gamma \delta} \nabla_{\gamma} g_{\alpha \delta} - \frac{1}{2} \nabla_{\alpha} \left( g^{\rho \sigma}\delta g_{\rho \sigma}\right)\,.
\end{align}
We first compute the spherical decomposition of the covariant derivatives 
of the  perturbation of the metric tensor \cite{Martel:2003ab}. 
The first covariant derivatives follow from
\begin{align}
    \nabla_{\gamma}p_{\alpha\beta}
    =
    \partial_{\gamma}p_{\alpha\beta}
    -
    \Gamma^{\delta}_{\gamma\alpha}p_{\delta\beta}
    -
    \Gamma^{\delta}_{\gamma\beta}p_{\delta\alpha}
    .
\end{align}
Using Eq.~\eqref{eq:chr_sym_spherical_sym}, we then have
\begin{subequations}
\label{eq:first_der_metric_pert}
\begin{align}
    \nabla_{c}p_{ab}
    =&
    \partial_{c}p_{b}
    -
    \Gamma^{d}_{ca}p_{db}
    -
    \Gamma^{d}_{cb}p_{da}
    \nonumber\\
    =&
    D_cp_{ab}
    ,\\
    \nabla_Cp_{ab}
    =&
    \partial_{C}p_{ab}
    -
    \Gamma^{D}_{Ca}p_{Db}
    -
    \Gamma^{D}_{Cb}p_{Da}
    \nonumber\\
    =&
    D_Cp_{ab}
    -
    \frac{2}{r}r_{(a}p_{b)C}
    ,\\
    \nabla_cp_{aB}
    =&
    \partial_{c}p_{aB}
    -
    \Gamma^{d}_{ca}p_{dB}
    -
    \Gamma^{D}_{cB}p_{Da}
    \nonumber\\
    =&
    D_cp_{aB}
    -
    \frac{1}{r}r_cp_{aB}
    ,\\
    \nabla_Cp_{aB}
    =&
    \partial_{C}p_{aB}
    -
    \Gamma^{D}_{Ca}p_{DB}
    -
    \Gamma^{D}_{CB}p_{Da}
    -
    \Gamma^{d}_{CB}p_{da}
    \nonumber\\
    =&
    D_Cp_{aB}
    -
    \frac{1}{r}r_ap_{BC}
    +
    r r^d \Omega_{BC} p_{da}
    ,\\
    \nabla_cp_{AB}
    =&
    \partial_{c}p_{AB}
    -
    \Gamma^{D}_{cA}p_{DB}
    -
    \Gamma^{D}_{cB}p_{DA}
    \nonumber\\
    =&
    D_cp_{AB}
    -
    \frac{2}{r}r_cp_{AB}
    ,\\
    \nabla_Cp_{AB}
    =&
    \partial_{C}p_{AB}
    -
    \Gamma^{D}_{CA}p_{DB}
    -
    \Gamma^{d}_{CA}p_{dB}
    -
    \Gamma^{D}_{CB}p_{DA}
    -
    \Gamma^{d}_{CB}p_{dA}
    \nonumber\\
    =&
    D_{C}p_{AB}
    +
    2rr^d\Omega_{C(A}p_{B)d}
    .
\end{align}
\end{subequations}

It then follows that
\begin{subequations}
\begin{align}
    v_a
    =&
    g^{\gamma\delta}\nabla_{\gamma}p_{\delta a}
    -
    \frac{1}{2}\nabla_a\left(g^{\gamma\delta}p_{\gamma\delta}\right)
    \nonumber\\
    =&
    \alpha^{cd}\nabla_{c}p_{da}
    +
    \frac{1}{r^2}\Omega^{CD}\nabla_{C}p_{Da}
    -
    \frac{1}{2}\nabla_ap
    \nonumber\\
    =&
    \alpha^{cd}D_{c}p_{da}
    +
    \frac{1}{r^2}\Omega^{CD}
    \left(
        D_Cp_{Da}
        -
        \frac{1}{r}r_ap_{CD}
        +
        rr^d\Omega_{CD}p_{ad}
    \right)
    -
    \frac{1}{2}D_ap
    \nonumber\\
    =&
    D_{c}p_{a}{}^c
    +
    \frac{1}{r^2}D_Cp_{a}{}^C
    -
    \frac{1}{r^3}r_ap_{C}{}^C
    +
    \frac{2}{r}r^dp_{ad}
    -
    \frac{1}{2}D_ap
    \\
    %%%%%%%%%%%%%%%%%%%%%%%%%%%%%%%%%
    v_A
    =&
    g^{\gamma\delta}\nabla_{\gamma}p_{\delta A}
    -
    \frac{1}{2}\nabla_A\left(g^{\gamma\delta}p_{\gamma\delta}\right)
    \nonumber\\
    =&
    \alpha^{cd}\nabla_{c}p_{d A}
    +
    \frac{1}{r^2}\Omega^{CD}\nabla_{C}p_{D A}
    -
    \frac{1}{2}\nabla_Ap
    \nonumber\\
    =&
    \alpha^{cd}D_{c}p_{d A}
    -
    \frac{1}{r}r^dp_{d A}
    +
    \frac{1}{r^2}\Omega^{CD}D_{C}p_{D A}
    +
    \frac{2}{r}r^d\Omega^{CD}\Omega_{C(D}p_{A)d}
    -
    \frac{1}{2}D_Ap
    \nonumber\\
    =&
    D_{c}p_{A}{}^c
    +
    \frac{1}{r^2}D_{C}p_{A}{}^C
    +
    \frac{2}{r}r^dp_{Ad}
    -
    \frac{1}{2}D_Ap
\end{align}
\end{subequations}
We also have
\begin{subequations}
\begin{align}
    \nabla_av_b
    =&
    \partial_av_b
    -
    \Gamma^c_{ab}v_c
    \nonumber\\
    =&
    D_av_b
    \\
    %%%%%%%%%%%%%%%%%%%%%%%%%%%%%%%%%%%%%%
    \nabla_av_B
    =&
    \partial_av_B
    -
    \Gamma^C_{aB}v_C
    \nonumber\\
    =&
    D_av_B
    -
    \frac{1}{r}r_av_B
    ,\\
    %%%%%%%%%%%%%%%%%%%%%%%%%%%%%%%%%%%%%%
    \nabla_Av_b
    =&
    \partial_Av_b
    -
    \Gamma^C_{Ab}v_C
    \nonumber\\
    =&
    D_Av_b
    -
    \frac{1}{r}r_bv_A
    ,\\
    %%%%%%%%%%%%%%%%%%%%%%%%%%%%%%%%%%%%%%
    \nabla_Av_B
    =&
    \partial_Av_B
    -
    \Gamma^C_{AB}v_C
    -
    \Gamma^c_{AB}v_c
    \nonumber\\
    =&
    D_Av_B
    +
    \Omega_{AB} r r^cv_c
    .
\end{align}
\end{subequations}

We compute the second derivative of the perturbed metric tensor using
\begin{align}
    \nabla_{\delta}\nabla_{\gamma}p_{\alpha\beta}
    =&
    \partial_{\delta}\left(\nabla_{\gamma}p_{\alpha\beta}\right)
    -
    \Gamma^{\rho}_{\delta\gamma}\left(\nabla_{\rho}p_{\alpha\beta}\right)
    -
    \Gamma^{\rho}_{\delta\alpha}\left(\nabla_{\gamma}p_{\rho\beta}\right)
    -
    \Gamma^{\rho}_{\delta\beta}\left(\nabla_{\gamma}p_{\alpha\rho}\right)
    .
\end{align}
To compute the perturbation of the Ricci tensor in spherical symmetry,
all we have to compute are
\begin{align}
    \nabla_c\nabla_d p_{\alpha\beta}
    ,\qquad
    \nabla_C\nabla_D p_{\alpha\beta}
    .
\end{align}
For $\alpha=a,\beta=b$, we have
\begin{subequations}
\label{eq:scd_der_metric_pert_ab}
\begin{align}
    \nabla_d\nabla_cp_{ab}
    =&
    \partial_{d}\left(\nabla_{c}p_{ab}\right)
    -
    \Gamma^{p}_{dc}\left(\nabla_{p}p_{ab}\right)
    -
    \Gamma^{p}_{da}\left(\nabla_{c}p_{pb}\right)
    -
    \Gamma^{p}_{db}\left(\nabla_{c}p_{ap}\right)
    \nonumber\\
    =&
    D_dD_cp_{ab}
    ,\\
    %%%%%%%%%%%%%%%%%%%%%%%%%%%%%%%%%%%%%%%%
    \nabla_D\nabla_Cp_{ab}
    =&
    \partial_{D}\left(\nabla_{C}p_{ab}\right)
    -
    \Gamma^{P}_{DC}\left(\nabla_{P}p_{ab}\right)
    -
    \Gamma^{p}_{DC}\left(\nabla_{p}p_{ab}\right)
    -
    \Gamma^{P}_{Da}\left(\nabla_{C}p_{Pb}\right)
    -
    \Gamma^{P}_{Db}\left(\nabla_{C}p_{aP}\right)
    \nonumber\\
    =&
    D_D\left(\nabla_Cp_{ab}\right)
    +
    r r^p\Omega_{CD} \nabla_pp_{ab}
    -
    \frac{2}{r}r_{(a}\nabla_{|C|}p_{b)D}
    \nonumber\\
    =&
    D_D\left(
        D_Cp_{ab}
        -
        \frac{2}{r}r_{(a}p_{b)C}
    \right)
    +
    r r^p\Omega_{CD} D_pp_{ab}
    \nonumber\\
    &
    -
    \frac{2}{r}r_{(a}\left(
        D_{|C|}p_{b)D}
        -
        \frac{1}{r}r_{b)}p_{CD}
        +
        p_{b)p}rr^p\Omega_{CD}
    \right)
    \nonumber\\
    =&
    D_DD_Cp_{ab}
    -
    \frac{2}{r}r_{(a}D_{|D|}p_{b)C}
    -
    \frac{2}{r}r_{(a}D_{|C|}p_{b)D}
    \nonumber\\
    &
    +
    \frac{2}{r^2}r_{a}r_{b}p_{CD}
    +
    rr^p\Omega_{CD}
    \left(
        D_pp_{ab}
        -
        \frac{2}{r}r_{(a}p_{b)p}
    \right)
\end{align}
\end{subequations}

For $\alpha=a,\beta=B$, we have
\begin{subequations}
\label{eq:scd_der_metric_pert_aB}
\begin{align}
    \nabla_d\nabla_cp_{aB}
    =&
    \partial_d\left(\nabla_cp_{aB}\right)
    -
    \Gamma^p_{dc}\left(\nabla_pp_{aB}\right)
    -
    \Gamma^p_{da}\left(\nabla_cp_{aB}\right)
    -
    \Gamma^P_{dB}\left(\nabla_cp_{aP}\right)
    \nonumber\\
    =&
    D_d\left(\nabla_cp_{aB}\right)
    -
    \frac{1}{r}r_d\nabla_cp_{aB}
    \nonumber\\
    =&
    D_d\left(D_cp_{aB} - \frac{1}{r}r_cp_{aB}\right)
    -
    \frac{1}{r}r_d\left(D_cp_{aB} - \frac{1}{r}r_cp_{aB}\right)
    \nonumber\\
    =&
    D_dD_cp_{aB} 
    + 
    \frac{1}{r^2}r_{c}r_{d}p_{aB}
    - 
    \frac{1}{r}r_{cd}p_{aB}
    - 
    \frac{1}{r}r_{c}D_dp_{aB}
    -
    \frac{1}{r}r_dD_cp_{aB} 
    +
    \frac{1}{r^2}r_cr_dp_{aB}
    \nonumber\\
    =&
    D_dD_cp_{aB} 
    - 
    \frac{2}{r}r_{(c}D_{d)}p_{aB}
    + 
    \left(
        \frac{2}{r^2}r_{c}r_{d}
        - 
        \frac{1}{r}r_{cd}
    \right)
    p_{aB}
    ,\\
    %%%%%%%%%%%%%%%%%%%%%%%%%%%%%%%%%%%%%%%
    \nabla_D\nabla_Cp_{aB}
    =&
    \partial_D\left(\nabla_Cp_{aB}\right)
    -
    \Gamma^P_{DC}\left(\nabla_Pp_{aB}\right)
    -
    \Gamma^p_{DC}\left(\nabla_pp_{aB}\right)
    -
    \Gamma^P_{Da}\left(\nabla_Cp_{PB}\right)
    \nonumber\\
    &
    -
    \Gamma^P_{DB}\left(\nabla_Cp_{aP}\right)
    -
    \Gamma^p_{DB}\left(\nabla_Cp_{ap}\right)
    \nonumber\\
    =&
    D_D\left(\nabla_Cp_{aB}\right)
    -
    \frac{1}{r}r_a\nabla_Cp_{DB}
    +
    rr^p \Omega_{CD} \nabla_pp_{aB}
    +
    rr^p\Omega_{BD}\nabla_Cp_{ap}
    \nonumber\\
    =&
    D_D\left(D_Cp_{aB} - \frac{1}{r}r_ap_{BC} + r r^d\Omega_{BC}p_{ad}\right)
    -
    \frac{1}{r}r_a\left(
        D_Cp_{BD}
        +
        2rr^d\Omega_{C(B}p_{D)d}
    \right)
    \nonumber\\
    &
    +
    rr^p \Omega_{CD}\left(
        D_pp_{aB}
        -
        \frac{1}{r}r_pp_{aB}
    \right)
    +
    rr^p\Omega_{BD}\left(
        D_Cp_{ap}
        -
        \frac{2}{r}r_{(a}p_{p)C}
    \right)
    \nonumber\\
    =&
    D_DD_Cp_{aB}
    -
    \frac{2}{r}r_aD_{(C}p_{D)B}
    +
    r r^p\left(
        \Omega_{CD}D_pp_{aB}
        -
        \frac{1}{r}r_a\Omega_{BC}p_{pD}
    \right)
    \nonumber\\
    &
    +
    2rr^p\left(
        \Omega_{B(C}D_{D)}p_{ap}
        -
        \frac{1}{r}r_p\Omega_{D(B}p_{C)a}
        -
        \frac{1}{r}r_a\Omega_{D(B}p_{C)p}
    \right)
    .
\end{align}
\end{subequations}

For $\alpha=A,\beta=B$, we have
\begin{subequations}
\label{eq:scd_der_metric_pert_AB}
\begin{align}
    \nabla_d\nabla_cp_{AB}
    =&
    \partial_d\left(\nabla_cp_{AB}\right)
    -
    \Gamma^p_{dc}\left(\nabla_pp_{AB}\right)
    -
    \Gamma^P_{dA}\left(\nabla_cp_{PB}\right)
    -
    \Gamma^P_{dB}\left(\nabla_cp_{PA}\right)
    \nonumber\\
    =&
    D_d\left(\nabla_cp_{AB}\right)
    -
    \frac{2}{r}r_d\left(\nabla_cp_{AB}\right)
    \nonumber\\
    =&
    D_dD_cp_{AB} 
    -
    \frac{2}{r}r_{cd}p_{AB}
    +
    \frac{6}{r^2}r_cr_dp_{AB}
    -
    \frac{4}{r}r_{(c}D_{d)}p_{AB}
    ,\\
    %%%%%%%%%%%%%%%%%%%%%%%%%%%%%%%%%
    \nabla_D\nabla_Cp_{AB}
    =&
    \partial_D\left(\nabla_Cp_{AB}\right)
    -
    \Gamma_{DC}^P\left(\nabla_Pp_{AB}\right)
    -
    \Gamma_{DC}^p\left(\nabla_pp_{AB}\right)
    \nonumber\\
    &
    -
    \Gamma_{DA}^P\left(\nabla_Cp_{PB}\right)
    -
    \Gamma_{DA}^p\left(\nabla_Cp_{pB}\right)
    -
    \Gamma_{DB}^P\left(\nabla_Cp_{PA}\right)
    -
    \Gamma_{DB}^p\left(\nabla_Cp_{pA}\right)
    \nonumber\\
    =&
    D_D\left(\nabla_Cp_{AB}\right)
    +
    \Omega_{CD}rr^p\left(\nabla_pp_{AB}\right)
    +
    2\Omega_{D(A}rr^p\left(\nabla_{|C|}p_{B)p}\right)
    \nonumber\\
    =&
    D_D\left(D_Cp_{AB} + 2rr^d\Omega_{C(A}p_{B)d}\right)
    \nonumber\\
    &
    +
    rr^p\Omega_{CD}\left(
        D_pp_{AB}
        -
        \frac{1}{r}r_pp_{AB}
    \right)
    \nonumber\\
    &
    +
    2rr^p\Omega_{D(A}\left(
        D_{|C|}p_{B)p}
        -
        \frac{1}{r}r_{|p|}p_{B)C}
        +
        rr^q\Omega_{B)C}p_{pq}
    \right)
    \nonumber\\
    =&
    D_DD_Cp_{AB}
    +
    2rr^p\left(
        \Omega_{C(A}D_{|D|}p_{B)p}
        +
        \Omega_{D(A}D_{|C|}p_{B)p}
    \right)
    \nonumber\\
    &
    -
    2r^pr_p\left(
        \Omega_{CD}p_{AB}
        +
        \Omega_{D(A}p_{B)C}
    \right)
    +
    2r^2r^pr^q\Omega_{D(A}\Omega_{B)C}p_{pq}
    +
    rr^p\Omega_{CD}D_pp_{AB}
\end{align}
\end{subequations}
These expressions match those in Martel's PhD thesis \cite{Martel:2003ab}.

We can now compute the covariant wave operator acting on the perturbed metric
\begin{align}
    g^{\gamma\delta}\nabla_{\gamma}\nabla_{\delta}p_{ab}
    =&
    \left(
        \alpha^{cd}\nabla_c \nabla_d
        +
        \frac{1}{r^2}\Omega^{CD}\nabla_C \nabla_D
    \right)
    p_{ab}
    \nonumber\\
    =&
    \left(
        D_cD^c
        +
        \frac{1}{r^2}D_CD^C
    \right)
    p_{ab}
    -
    \frac{4}{r^3}r_{(a}D_{|C|}p_{b)}{}^C
    \nonumber\\
    &
    +
    \frac{2}{r}r^c\left(
        D_cp_{ab}
        -
        \frac{2}{r}r_{(a}p_{b)c}
    \right)
    +
    \frac{2}{r^4}r_{(a}r_{b)}\Omega^{CD}p_{CD}
    \\
    %%%%%%%%%%%%%%%%%%%%%%%%%%%%%%%%%%%%
    g^{\gamma\delta}\nabla_{\gamma}\nabla_{\delta}p_{aB}
    =&
    \left(
        \alpha^{cd}\nabla_c \nabla_d
        +
        \frac{1}{r^2}\Omega^{CD}\nabla_C \nabla_D
    \right)
    p_{aB}
    \nonumber\\
    =& 
    \left(
        D_cD^c
        +
        \frac{1}{r^2}D_CD^C
    \right)
    p_{aB}
    -
    \frac{1}{r}
    \left(
        \frac{1}{r}r_cr^c
        +
        r_c{}^c
    \right)
    p_{aB}
    -
    \frac{4}{r^2}r_ar^cp_{Bc}
    \nonumber\\
    &
    -
    \frac{2}{r^3}r_aD_Cp_{B}{}^C
    +
    \frac{2}{r}r^cD_Bp_{ac}
    \\
    %%%%%%%%%%%%%%%%%%%%%%%%%%%%%%%%%%%%
    g^{\gamma\delta}\nabla_{\gamma}\nabla_{\delta}p_{AB}
    =&
    \left(
        \alpha^{cd}\nabla_c \nabla_d
        +
        \frac{1}{r^2}\Omega^{CD}\nabla_C \nabla_D
    \right)
    p_{AB}
    \nonumber\\
    =& 
    \left(
        D_cD^c
        +
        \frac{1}{r^2}D_CD^C
    \right)
    p_{AB}
    -
    \frac{2}{r}\left(
        r_c{}^c
        +
        r^cD_c
    \right)
    p_{AB}
    \nonumber\\
    &
    +
    \frac{4}{r}r^cD_{(A}p_{B)c}
    +
    2\Omega_{AB}r^cr^dp_{cd}
    .
\end{align}

We next consider the covariant components of the Riemann tensor contracted
with the perturbed metric.
\begin{subequations}
\begin{align}
    R_{a}{}^{\gamma}{}_b{}^{\delta}p_{\gamma\delta}
    =&
    R_{a}{}^{c}{}_b{}^{d}p_{cd}
    +
    R_{a}{}^{C}{}_b{}^{D}p_{CD}
    \nonumber\\
    =&
    \frac{1}{2}\mathcal{R}
    \left(\alpha_{ab}\alpha^{cd} - \delta^d_a\delta^c_b\right)
    p_{cd}
    -
    \frac{1}{r^3}\left(r_{ab}\right)\Omega^{CD}p_{CD}
    \\
    %%%%%%%%%%%%%%%%%%%%
    R_a{}^{\gamma}{}_B{}^{\delta}p_{\gamma\delta}
    =&
    R_a{}^C{}_B{}^dp_{Cd}
    \nonumber\\
    =&
    \left(\frac{1}{r}r_a^d\right)p_{Bd}
    \\
    %%%%%%%%%%%%%%%%%%%%
    R_A{}^{\gamma}{}_B{}^{\delta}p_{\gamma\delta}
    =&
    R_A{}^{c}{}_B{}^{d}p_{cd}
    +
    R_A{}^{C}{}_B{}^{D}p_{CD}
    \nonumber\\
    =&
    -
    \Omega_{AB}\left(r r^{cd}\right)p_{cd}
    +
    \frac{1}{r^2}\left(
        1
        -
        r_ar^a
    \right)
    \left(\Omega_{AB}\Omega^{CD} - \delta^D_A\delta^C_B\right)
    p_{CD}
    .
\end{align}
\end{subequations}
Finally, we consider the covariant components of the Ricci tensor
contracted with the perturbed metric
\begin{subequations}
\begin{align}
    R^{\gamma}{}_{(a}p_{b)\gamma}
    =&
    R^c{}_{(a}p_{b)c}
    \nonumber\\
    =&
    \frac{1}{2}\mathcal{R} p_{ab}
    -
    \frac{2}{r}r^c_{(a}p_{b)c}
    \\
    %%%%%%%%%%%%%%%%%%%%%%%%%%%%
    R^{\gamma}{}_{(a}p_{B)\gamma}
    =&
    \frac{1}{2}\left(
        R^c{}_{a}p_{Bc}
        +
        R^C{}_{B}p_{aC}
    \right)
    \nonumber\\
    =&
    -
    \frac{1}{r}r_a^cp_{Bc}
    +
    \left(
        \frac{1}{4}\mathcal{R}
        +
        \frac{1}{2r^2}
        \left(
            1
            -
            r_cr^c
            -
            rr_c^c
        \right)
    \right)
    p_{aB}
    \\
    %%%%%%%%%%%%%%%%%%%%%%%%%%%%
    R^{\gamma}{}_{(A}p_{B)\gamma}
    =&
    R^{C}{}_{(A}p_{B)C}
    \nonumber\\
    =&
    \frac{1}{r^2}
    \left(
        1
        -
        r_cr^c
        -
        rr_c^c
    \right)
    p_{AB}
    .
\end{align}
\end{subequations}
Alternatively, we could make use of the Einstein equations are write this as
\begin{subequations}
\begin{align}
    R^{\gamma}{}_{(a}p_{b)\gamma}
    =&
    \kappa \hat{T}^c{}_{(a}p_{b)c}
    \\
    %%%%%%%%%%%%%%%%%%%%%%%%%%%%
    R^{\gamma}{}_{(a}p_{B)\gamma}
    =&
    \frac{\kappa}{2}\left(
        \hat{T}^c{}_{a}p_{Bc}
        +
        \hat{T}^C{}_{B}p_{aC}
    \right)
    \nonumber\\
    =&
    \frac{\kappa}{2}\left(
        \hat{T}^c{}_{a}p_{Bc}
        -
        \left(\mathcal{E} - \mathcal{P}\right)
        p_{aB}
    \right)
    \\
    %%%%%%%%%%%%%%%%%%%%%%%%%%%%
    R^{\gamma}{}_{(A}p_{B)\gamma}
    =&
    \frac{\kappa}{2}\left(\mathcal{E} - \mathcal{P}\right)
    p_{AB}
    .
\end{align}
\end{subequations}
%---------------------------------------------------------------------------
\chapter{Axial and polar spherical harmonic decomposition of the Einstein equations
\label{sec:axial_decomposition_tensor_eom}}

\section{Axial perturbations}

We only need to consider the components $\delta R_{aB}$ and $\delta R_{AB}$.
As we reviewed in Sec.~\ref{sec:pert_spherically_symmetric_spacetime}, 
as first shown in \cite{Martel:2005ir}, 
we can work in the Regge-Wheeler gauge, and then promote the variables to
gauge-invariant ones at the end.
We also drop the $\ell,m$ labels to make the equations less cluttered.
We use $\dot{=}$ to indicate that we are dropping all terms that are
zero for an axial perturbation in Regge-Wheeler gauge.
The only nonzero component of the metric perturbation then is
\begin{align}
    p_{aB}
    \dot{=}
    h_a S_B
    .
\end{align}

%---------------------------------------------------------------------------
\subsection{Computing the $aB$ component of the Ricci tensor}
We first look at
\begin{align}
    g^{\gamma\delta}\nabla_{\gamma}\nabla_{\delta}p_{aB}
    \dot{=}&
    \left(
        D_cD^c + \frac{1}{r^2}D_CD^C
    \right)
    S_B
    h_a
    -
    \frac{1}{r}
    \left(
        \frac{1}{r}r_cr^c
        +
        r_c^c 
    \right)
    S_Bh_a
    -
    \frac{4}{r^2}r_ar^ch_cS_B
    \nonumber\\
    =&
    \left(
        \left(
            D_cD^c
            +
            \frac{1}{r^2}\left(1-\ell\left(\ell+1\right)\right)
            -
            \frac{1}{r^2}r_cr^c
            -
            \frac{1}{r}r_c{}^c
        \right)
        h_{a}
        -
        \frac{4}{r^2}r_ar_c
        h^{c}
    \right)
    S_B
    .
\end{align}
We next look at
\begin{align}
    \nabla_av_B
    +
    \nabla_Bv_a
    \dot{=}&
    \left(
        D_a
        -
        \frac{2}{r}r_a
    \right)
    v_B
    \nonumber\\
    \dot{=}&
    \left(D_a-\frac{2}{r}r_a\right)
    \left(
        D_ch^c
        +
        \frac{2}{r}r_ch^c
    \right)
    S_B
    \nonumber\\
    =&
    \left(
        D_aD_ch^c
        +
        \frac{2}{r}r_cD_ah^c
        -
        \frac{2}{r}r_aD_ch^c
        +
        \frac{2}{r}r_{ac}h^c
        -
        \frac{6}{r^2}r_ar_ch^c
    \right)
    S_B
    .
\end{align}
For the Riemann and Ricci tensor components, we have
\begin{align}
    R_a{}^{\gamma}{}_B{}^{\delta}p_{\gamma\delta}
    =&
    \left(\frac{1}{r}r_{ac}\right)h^c S_B
    \\
    %%%%%%%%%%%%%%%%%%%%%%
    R^{\gamma}{}_{(a}p_{B)\gamma}
    =&
    \frac{1}{2}\left(
        \kappa\hat{T}_{ac}h^c
        +
        \frac{1}{r^2}\left(1-r_cr^c-rr_c^c\right)h_a
    \right)
    S_B
    .
\end{align}
We have substituted the trace-reverse of the stress-energy
tensor $\hat{T}_{ab}$ for $R_{ab}$, while directly writing out
the expression for $R_{AB}$, so our formulas will match those in
\cite{Martel:2005ir}\footnote{Their Eq. 5.8, although note that
those authors assume the background is vacuum so $R_{\mu\nu}=0$.}. 
Using Eq.~\eqref{eq:general_pert_Ricci_tensor}, 
and promoting everything to gauge-invariant quantities, 
we are left with
\begin{align}
    \label{eq:axial_pert_aB_comp_ricci_tensor}
    \boxed{
        \delta R_{aB}
        =
        \frac{1}{2}
        \left(
            -
            \left(
                D_cD^c
                -
                \frac{1}{r^2}\ell\left(\ell+1\right)
            \right)
            \tilde{h}_a
            +
            D_aD_c\tilde{h}^c
            +
            \frac{4}{r}r_{[c}D_{a]}\tilde{h}^c
            -
            \frac{2}{r^2}r_ar_c
            \tilde{h}^c
            +
            \kappa \hat{T}_{ac}\tilde{h}^c
        \right)
        S_B
        .
    }
\end{align}
We can now write down the tensor equations of motion,
\begin{align}
    \delta R_{aB}
    \dot{=}
    \kappa
    \left(
        \delta T_{aB}
        -
        \frac{1}{2}\delta g_{aB} T
    \right)
    .
\end{align}
%---------------------------------------------------------------------------
\subsection{Computing the $AB$ component of the Ricci tensor}
We first look at
\begin{align}
    g^{\gamma\delta}\nabla_{\gamma}\nabla_{\delta}p_{AB}
    \dot{=}&
    \frac{4}{r}r_c
    \Omega^{CD}\Omega_{C(A}D_{|D|}S_{B)}
    h^c 
    \nonumber\\
    =&
    \frac{4}{r}r_ch^c S_{AB}
    .
\end{align}
We next look at
\begin{align}
    \nabla_Av_B
    +
    \nabla_Bv_A
    \dot{=}&
    D_Av_B + D_Bv_A
    \nonumber\\
    \dot{=}&
    \left(D_AS_B + D_BS_A\right)
    \left(D_ch^c + \frac{2}{r}r_ch^c\right)
    \nonumber\\
    =&
    2
    \left(D_ch^c + \frac{2}{r}r_ch^c\right)
    S_{AB}
\end{align}
where, $S_{AB} \equiv D_{(A} S_{B)}$.
The other terms $R_{A}{}^{\gamma}{}_{B}{}^{\gamma}p_{\gamma\delta}$
and $R^{\gamma}{}_{(A}p_{B)\gamma}$ are zero for axial perturbations
in the Regge-Wheeler gauge.
Using Eq.~\eqref{eq:general_pert_Ricci_tensor}, 
and promoting everything to gauge-invariant quantities, we have
\begin{align}
    \label{eq:axial_pert_AB_comp_ricci_tensor}
    \boxed{
        \delta R_{AB}
        =
        D_c\tilde{h}^c S_{AB}
        .
    }
\end{align}
Note that the axial perturbation of the $AB$ component of the
Einstein and Ricci tensors are the same in the Regge-Wheeler gauge
\begin{align}
    \delta G_{AB}
    \dot{=}
    \delta R_{AB}
    .
\end{align}
This is why our expression Eq.~\eqref{eq:axial_pert_AB_comp_ricci_tensor}
matches Eq (5.9) of \cite{Martel:2005ir}.

%---------------------------------------------------------------------------
\subsection{Computing the $a$ component of the Bianchi identity}
    There are no axial perturbations of this component.
%---------------------------------------------------------------------------
\subsection{Computing the $A$ component of the Bianchi identity}

We next consider the divergence of the stress-energy tensor
(see Sec.~\eqref{sec:pert_stress_energy_tensor_ss}). We first look at
\begin{align}
    D_cP^{c}_{A}
    +
    \frac{2}{r}r_cP^{c}_{A}
    +
    D_CP^{C}_{A}
    \dot{=}&
    S^A
    \left(
        D_cH^c
        +
        \frac{2}{r}r_cH^c
    \right)
    +
    \frac{1}{r^2}D_CS^{C}_{A} H_2
    \nonumber\\
    =&
    S^A\left(
        \frac{1}{r^2}D_c\left(r^2H^c\right)
        +
        \frac{1}{r^2}
        \left(
            1
            -
            \frac{1}{2}\ell\left(\ell+1\right)
        \right)
        H_2
    \right)
    .
\end{align}
Next, looking at the metric perturbations, we have
\begin{align}
    \left(
        \frac{1}{2}D_Ap^c_c
        -
        \frac{1}{r}r_cp^c_A
    \right)
    \mathcal{P}
    -
    \left(
        \frac{1}{2}D_Ap^c_b
        -
        \frac{1}{r}r_bp^c_A
    \right)
    T^b_c 
    \dot{=}&
    S^A
    \left(
        -
        \frac{1}{r}r_ch^c
        \mathcal{P}
        +
        \frac{1}{r}r_bh^cT^b_c
    \right)
    .
\end{align}
Promoting $h_a$ to its gauge invariant counterpart $\tilde{h}_a$,
we conclude that the axial matter equations of motion are 
\begin{align}
    \label{eq:axial_comp_A_se_eom}
    \boxed{
        \frac{1}{r^2}D_c\left(r^2H^c\right)
        -
        \frac{\left(\ell-1\right)\left(\ell+2\right)}{2r^2}
        H_2
        +
        \frac{1}{r}r_c\tilde{h}^bT^c_b
        -
        \frac{1}{r}r_c\tilde{h}^c\mathcal{P}
        =
        0
        .
    }
\end{align}

%==============================================================================
\section{Polar spherical harmonic decomposition of the Einstein equations
\label{sec:polar_decomposition_tensor_eom}}
    We need to consider the components $\delta R_{ab}$, $\delta R_{aB}$,
and $\delta R_{AB}$. 
As we review in Sec.~\ref{sec:pert_spherically_symmetric_spacetime}, 
we can work in the Regge-Wheeler
gauge, and promote all variables to their gauge-invariant
counterparts at the end of the calculation.
We now use $\dot{=}$ to indicate that we only keep terms that are nonzero
in a polar decomposition in Regge-Wheeler gauge.
We work in a spherical harmonic basis and drop the $\ell,m$ labels
to make the expressions less cluttered.
The only nonzero components of the metric perturbation are
\begin{subequations}
\begin{align}
    p_{ab}
    \dot{=}&
    h_{ab} Y 
    ,\\
    p_{AB}
    \dot{=}&
    r^2 k \Omega_{AB} Y
    .
\end{align}
\end{subequations}
%---------------------------------------------------------------------------
\subsection{Computing the $ab$ components of the Ricci tensor}
    We first look at
\begin{align}
    g^{\gamma\delta}\nabla_{\gamma}\nabla_{\delta}p_{ab}
    \dot{=}&
    \left( 
        D_cD^c
        +
        \frac{1}{r^2}D_CD^C
    \right)
    h_{ab}Y
    +
    \frac{2}{r}r^c
    \left(
        D_ch_{ab}
        -
        \frac{2}{r}r_{(a}h_{b)c}
    \right)
    Y
    +
    \frac{4}{r^2}r_{(a}r_{b)}k Y
    \nonumber\\
    =&
    \left(
        \left( 
            D_cD^c
            -
            \frac{\ell\left(\ell+1\right)}{r^2}
        \right)
        h_{ab}
        +
        \frac{2}{r}r^c
        \left(
            D_ch_{ab}
            -
            \frac{2}{r}r_{(a}h_{b)c}
        \right)
        +
        \frac{4}{r^2}r_{(a}r_{b)}k
    \right)
    Y
    .
\end{align}
We next look at (we can symmetrize later)
\begin{align}
    \nabla_av_b
    \dot{=}&
    D_a\left(
        D_ch_b{}^c
        -
        \frac{2}{r}r_bk
        +
        \frac{2}{r}r^ch_{bc}
        -
        \frac{1}{2}D_b\left(h + 2k\right)
    \right)
    Y
    \nonumber\\
    =&
    \Bigg(
        D_aD_ch_b{}^c
        +
        \frac{2}{r^2}r_ar_bk
        +
        \frac{1}{r}r^c\left(
            2D_{(a}h_{b)c}
            -
            D_ch_{ab}
        \right)
        \nonumber\\
        &
        -
        \frac{2}{r^2}r_ar^ch_{bc}
        +
        \frac{2}{r}\left(r_a^ch_{bc} + r^cD_ah_{bc}\right)
        -
        D_aD_b\left(\frac{1}{2}h + k\right)
    \Bigg)
    Y
    .
\end{align}
For the Riemann and Ricci tensor components, we have
\begin{align}
    R_{a}{}^{\gamma}{}_b{}^{\delta}p_{\gamma\delta}
    \dot{=}&
    \left(
        \frac{1}{2}\mathcal{R}
        \left(\alpha_{ab}\alpha^{cd} - \delta^c_a\delta^d_b\right)h_{cd}
        -
        \frac{2}{r}r_{ab}k
    \right)
    Y
    ,\\
    R^{\gamma}{}_{(a}p_{b)\gamma}
    \dot{=}&
    \left(
        \frac{1}{2}\mathcal{R}h_{ab}
        -
        \frac{2}{r}r^c_{(a}h_{b)c}
    \right)
    Y
    .
\end{align}
Using Eq.~\eqref{eq:general_pert_Ricci_tensor}, 
and promoting everything to gauge-invariant quantities,
we end up with
\begin{empheq}[box=\widefbox]{align}
    \label{eq:polar_pert_ab_comp_ricci_tensor}
    \delta R_{ab}
    \dot{=}&
    \Bigg(
        -
        \frac{1}{2}\left(
            D_cD^c 
            -
            \frac{\ell\left(\ell+1\right)}{r^2}
        \right)
        \tilde{h}_{ab}
        +
        D_{(a}D^c\tilde{h}_{b)c}
        -
        D_aD_b\left(\frac{1}{2}\tilde{h} + \tilde{k}\right)
        \nonumber\\
        &\;\;\;
        +
        \frac{2}{r}r^c\left(D_c\tilde{h}_{ab} + D_{(a}\tilde{h}_{b)c}\right)
        -
        \frac{2}{r}r_{(a}D_{b)}\tilde{k}
        +
        \frac{1}{2}\mathcal{R}\left(
            3\tilde{h}_{ab}
            -
            \alpha_{ab}\tilde{h}
        \right)
    \Bigg)
    Y
    .
\end{empheq}

%---------------------------------------------------------------------------
\subsection{Computing the $aB$ component of the Ricci tensor}
    We first look at
\begin{align}
    g^{\gamma\delta}\nabla_{\gamma}\nabla_{\delta}p_{aB}
    \dot{=}&
    \left(
        -
        \frac{2}{r}r_ak
        +
        \frac{2}{r}r^ch_{ac}
    \right)
    E_B
    .
\end{align}
We next look at 
\begin{align}
    \nabla_av_B
    +
    \nabla_Bv_a
    =&
    D_av_B
    +
    D_Bv_a
    -
    \frac{2}{r}r_av_B
    \nonumber\\
    \dot{=}&
    \left(
        D_ch_a{}^c
        -
        D_ak
        -
        D_ah
        +
        \frac{2}{r}r_ch_a{}^c
        +
        \frac{1}{r}r_ah
        -
        \frac{2}{r}r_ak
    \right)
    E_B
    .
\end{align}
The Riemann tensor components are zero
\begin{align}
    R_{a}{}^{\gamma}{}_B{}^{\delta}p_{\gamma\delta}
    \dot{=}&
    0
    .
\end{align}
The Ricci tensor components are also zero
\begin{align}
    R^{\gamma}{}_{(a}p_{B)\gamma}
    \dot{=}&
    0
    .
\end{align}
Using Eq.~\eqref{eq:general_pert_Ricci_tensor}, 
and promoting to gauge-invariant quantities, we have
\begin{align}
    \label{eq:polar_pert_aB_comp_ricci_tensor}
    \boxed{
        \delta R_{aB}
        \dot{=}
        \frac{1}{2}
        \left(
            D_c\tilde{h}_a{}^c
            -
            D_a\tilde{h}
            -
            D_a\tilde{k}
            +
            \frac{1}{r}r_a\tilde{h}
        \right)
        E_B
        .
    }
\end{align}

%---------------------------------------------------------------------------
\subsection{Computing the $AB$ components of the Ricci tensor}
    We first look at
\begin{align}
    g^{\gamma\delta}\nabla_{\gamma}\nabla_{\delta}p_{AB}
    \dot{=}&
    \left(
        D_cD^c
        +
        \frac{1}{r^2}D_CD^C
    \right)
    \Omega_{AB}r^2kY
    -
    \frac{2}{r}\left(r_c^c + r^cD_c\right)\Omega_{AB}r^2kY
    +
    2\Omega_{AB}r^cr^dh_{cd}Y
    \nonumber\\
    =&
    \left(
        \left(
            D_cD^c
            -
            \frac{\ell\left(\ell+1\right)}{r^2}
        \right)
        k
        +
        \frac{2}{r}r^cD_ck
        -
        \frac{2}{r^2}r_cr^ck
        +
        \frac{2}{r^2}r^cr^dh_{cd}
    \right)
    r^2\Omega_{AB}Y
    .
\end{align}
We next look at (we can symmetrize later) 
\begin{align}
    \nabla_Av_B
    =&
    D_Av_B
    +
    \Omega_{AB}rr^cv_c
    \nonumber\\
    =&
    -
    \frac{1}{2}h Z_{AB}
    \nonumber\\
    &
    +
    \left(
        \frac{1}{r}r^cD_dh_c{}^d
        -
        \frac{1}{r}r^cD_ch
        -
        \frac{2}{r}r^cD_ck
        +
        \frac{2}{r^2}r^cr^dh_{cd}
        -
        \frac{2}{r^2}r_cr^ck
        -
        \frac{\ell\left(\ell+1\right)}{2r^2}h
    \right)
    r^2\Omega_{AB}Y
    .
\end{align}
The Riemann tensor components are
\begin{align}
    R_{A}{}^{\gamma}{}_B{}^{\delta}p_{\gamma\delta}
    \dot{=}&
    \left(
        -
        \frac{1}{r}r^{cd}h_{cd}
        +
        \frac{1-r_ar^c}{r^2}k
    \right)
    r^2\Omega_{AB}
    Y
    .
\end{align}
The Ricci tensor components are also zero
\begin{align}
    R^{\gamma}{}_{(A}p_{B)\gamma}
    \dot{=}&
    \left(
        \frac{1-r_ar^a-rr^a_a}{r^2}k
    \right)
    r^2\Omega_{AB}Y
    .
\end{align}
Using Eq.~\eqref{eq:general_pert_Ricci_tensor}, 
and promoting to gauge-invariant quantities, we have
\begin{empheq}[box=\widefbox]{align}
    \label{eq:polar_pert_AB_comp_ricci_tensor}
    \delta R_{AB}
    =&
    -
    \frac{1}{2}\tilde{h}Z_{AB}
    \nonumber\\
    &
    +
    \Bigg(
        -
        \frac{1}{2}\left(D_cD^c - \frac{\ell\left(\ell+1\right)}{r^2}\right)
        \tilde{k}
        -
        \frac{3}{r}r^cD_c\tilde{k}
        -
        \left(\frac{1}{r^2}r^cr^c - \frac{1}{r}r^c_c\right)\tilde{k}
        \nonumber\\
        &
        \qquad
        +
        \frac{1}{r}r^cD_d\tilde{h}_c{}^d
        -
        \frac{1}{r}r^cD_c\tilde{h}
        +
        \frac{2}{r^2}r^cr^d\tilde{h}_{cd}
        -
        \frac{\ell\left(\ell+1\right)}{2r^2}\tilde{h}
    \Bigg)
    r^2\Omega_{AB}Y
    .
\end{empheq}
%---------------------------------------------------------------------------
\subsection{Computing the $a$ component of the Bianchi identity}
    We consider the divergence of the stress-energy tensor.
We first look at
\begin{align}
    D_cP^c_a 
    +
    D_CP^C_a
    + 
    \frac{2}{r}r_cP^c_a 
    -
    \frac{1}{r}r_aP^C_C
    =
    \left(
        D_cH^c_a
        +
        \frac{2}{r}r_cH^c_a
        -
        \frac{\ell\left(\ell+1\right)}{r^2}J_a
        -
        \frac{2}{r}r_aK
    \right)
    Y
    .
\end{align}
We next look at the metric terms 
\begin{align}
    -
    C^b_{ca}T^b_b
    +
    \left(
        C^c_{cb}
        -
        \frac{1}{r^4}r_bp^C_C
    \right)
    T^b_a
    +
    \frac{1}{r^3}r_ap^C_C\mathcal{P}
    =
    \left(
        -
        C^b_{ca}T^b_b
        +
        \left(
            C^c_{cb}
            -
            \frac{2}{r^2}r_bk
        \right)
        T^b_a
        +
        \frac{2}{r}r_ak\mathcal{P}
    \right)
    Y
    .
\end{align}
    Putting everything together, we have
\begin{align}
    \label{eq:polar_comp_a_se_eom}
    \boxed{
        \frac{1}{r^2}D_c\left(r^2H^c_a\right)
        -
        \frac{\ell\left(\ell+1\right)}{r^2}J_a
        -
        \frac{2}{r}r_aK
        -
        C^b_{ca}T^b_b
        +
        \left(
            C^c_{cb}
            -
            \frac{2}{r^2}r_bk
        \right)
        T^b_a
        +
        \frac{2}{r}r_ak\mathcal{P}
        =
        0
        .
    }
\end{align}
%---------------------------------------------------------------------------
\subsection{Computing the $A$ component of the Bianchi identity}
    We consider the divergence of the stress-energy tensor.
We first look at
\begin{align}
    D_cP^c_A
    +
    D_CP^C_A
    +
    \frac{2}{r}r_cP^c_A
    =
    \left(
        D_cJ^c
        +
        K
        -
        \frac{\left(\ell+2\right)\left(\ell-1\right)}{2}G
        +
        \frac{2}{r}r_cJ^c
    \right)
    E_A
    .
\end{align}
We next look at the metric terms
\begin{align}
    \frac{1}{2}D_Ap^c_c\mathcal{P}
    -
    \frac{1}{2}D_Ap^b_c T^c_b
    =
    \frac{1}{2}
    \left(
        h\mathcal{P}
        -
        T^b_ch^c_b
    \right)
    E_A
    .
\end{align}
Putting everything together, we have
\begin{align}
    \label{eq:polar_comp_A_se_eom}
    \boxed{
        \frac{1}{r^2}D_c\left(r^2J^c\right)
        +
        K
        -
        \frac{\left(\ell+2\right)\left(\ell-1\right)}{2}G
        +
        h\mathcal{P}
        -
        T^b_ch^c_b
        =
        0
        .
    }
\end{align}

\appendix
%===========================================================================
\chapter{Scalar, vector, and tensor spherical harmonics
\label{sec:scalar_vector_tensor_spherical_harmonics}}
We work on the unit two-sphere $\mathbb{S}^2$, with metric $\Omega_{AB}$,
Levi-Cevita tensor $\varepsilon_{AB}$, 
and metric compatible derivative $D_A$. 
The Ricci tensor is $R=+2$.
Our notation for the spherical harmonics follows that of
\cite{Nagar:2005ea}.

%---------------------------------------------------------------------------
\section{Scalar spherical harmonics}
The scalar spherical harmonics satisfy
\begin{align}
    \left(
        \Omega^{AB}D_AD_B
        +
        \ell\left(\ell+1\right)
    \right)
    Y^m_{\ell}
    =
    0
    ,
\end{align}
along with the following orthogonality relation
\begin{align}
    \int d\Omega Y^m_{\ell}Y^{m^{\prime}}_{\ell^{\prime}}
    =
    \delta_{\ell\ell^{\prime}}\delta_{mm^{\prime}}
    .
\end{align}
%---------------------------------------------------------------------------
\section{Vector spherical harmonics}
The polar and axial
vector spherical harmonics respectively are
\begin{align}
    \left[E^m_{\ell}\right]_A
    \equiv
    D_AY^m_{\ell}
    ,\qquad
    \left[S^m_{\ell}\right]_A
    \equiv
    \varepsilon_{BA}D^BY^m_{\ell}
    .
\end{align}
The vector spherical harmonics satisfy
\begin{align}
    \left(
        \Omega^{AB}D_AD_B
        +
        \left(-1+\ell\left(\ell+1\right)\right)
    \right)
    \left[V^m_{\ell}\right]_C
    =
    0
    ,
\end{align}
along with the following orthogonality relation
\begin{align}
    \int d\Omega 
        \left[V^m_{\ell}\right]_A
        \left[V^{m^{\prime}}_{\ell^{\prime}}\right]^A
    =
    \ell\left(\ell+1\right)
    \delta_{\ell\ell^{\prime}}\delta_{mm^{\prime}}
    .
\end{align}
The divergence of the polar and axial vector spherical harmonics
respectively are
\begin{align}
    D_A\left[E^m_{\ell}\right]^A
    =&
    D_AD^AY^m_{\ell}
    \nonumber\\
    =&
    -
    \ell\left(\ell+1\right)Y^m_{\ell}
    .\\
    D_A\left[S^m_{\ell}\right]^A
    =&
    \varepsilon_{BA}D^BD^AY^m_{\ell}
    \nonumber\\
    =&
    0
    .
\end{align}
%---------------------------------------------------------------------------
\section{Tensor spherical harmonics}
The polar and axial tensor spherical harmonics respectively are
\begin{align}
    \left[Z^m_{\ell}\right]_{AB}
    \equiv
    D_AD_BY^m_{\ell}
    +
    \frac{1}{2}\ell\left(\ell+1\right)
    \Omega_{AB}Y^m_{\ell}
    ,\qquad
    \left[S^m_{\ell}\right]_{AB}
    \equiv
    D_{(A}\left[S^m_{\ell}\right]_{B)}
    .
\end{align}
The tensor spherical harmonics satisfy
\begin{align}
    \left(
        \Omega^{AB}D_AD_B
        +
        \left(-2+\ell\left(\ell+1\right)\right)
    \right)
    \left[T^m_{\ell}\right]_{CD}
    =
    0
    ,
\end{align}
along with the following orthogonality relation
\begin{align}
    \int d\Omega 
        \left[T^m_{\ell}\right]_{AB}
        \left[T^{m^{\prime}}_{\ell^{\prime}}\right]^{AB}
    =
    \frac{1}{2}
    \left(\ell-1\right)\ell\left(\ell+1\right)\left(\ell+2\right)
    \delta_{\ell\ell^{\prime}}\delta_{mm^{\prime}}
    .
\end{align}
The polar and axial tensor spherical harmonics are both traceless
\begin{align}
    \Omega^{AB}\left[Z^m_{\ell}\right]_{AB}
    =&
    D_AD^AY^m_{\ell}
    +
    \ell\left(\ell+1\right)Y^m_{\ell}
    \nonumber\\
    =&
    0
    ,\\
    \Omega^{AB}\left[S^m_{\ell}\right]_{AB}
    =&
    D_A\left[S^m_{\ell}\right]^A
    \nonumber\\
    =&
    0
    .
\end{align}
The trace is captured by the scalar spherical harmonic $Y^m_{\ell}$,
which is sometimes denoted by \cite{Martel:2003ab}
\begin{align}
    \left[U^m_{\ell}\right]_{AB}
    \equiv
    \Omega_{AB} Y^m_{\ell}
    .
\end{align}
The divergence of the polar and axial tensor spherical harmonics
respectively are
\begin{align}
    D_A\left[Z^m_{\ell}\right]^{AB}
    =&
    D_AD^AD^BY^m_{\ell}
    +
    \frac{1}{2}\ell\left(\ell+1\right)D^BY^m_{\ell}
    \nonumber\\
    =&
    D^BD_AD^AY^m_{\ell}
    +
    R^B_CD^CY^m_{\ell}
    +
    \frac{1}{2}\ell\left(\ell+1\right)D^BY^m_{\ell}
    \nonumber\\
    =&
    D^BY^m_{\ell}
    -
    \frac{1}{2}\ell\left(\ell+1\right)D^BY^m_{\ell}
    \nonumber\\
    =&
    \left(
        1
        -
        \frac{1}{2}\ell\left(\ell+1\right)
    \right)
    \left[E^m_{\ell}\right]^B
    ,\\
    D_A\left[S^m_{\ell}\right]^{AB}
    =&
    \frac{1}{2}D_A\left(
        D^A\left[S^m_{\ell}\right]^B
        +
        D^B\left[S^m_{\ell}\right]^A
    \right)
    \nonumber\\
    =&
    \frac{1}{2}\left(
        \left(1-\ell\left(\ell+1\right)\right)\left[S^m_{\ell}\right]^B
        +
        D_BD_A\left[S^m_{\ell}\right[^A
        +
        R^B_C\left[S^m_{\ell}\right]^C
    \right)
    \nonumber\\
    =&
    \left(
        1
        -
        \frac{1}{2}\ell\left(\ell+1\right)
    \right)
    \left[S^m_{\ell}\right]^B
    .
\end{align}
We have used that $R=2$ and $R_{AB} = (R/2)\Omega_{AB}=\Omega_{AB}$.

%==============================================================================
\bibliography{jripley_notes_bib}
\bibliographystyle{alpha}
%==============================================================================
\end{document}
