%%%%%%%%%%%%%%%%%%%%%%%%%%%%%%%%%%%%%%%%%%%%%%%%%%%%%%%%%%%%%%%%%%%%%%%%%%%%%%%
\documentclass[12pt]{report}
\usepackage[utf8]{inputenc}
\usepackage[T1]{fontenc}
%%%%%%%%%%%%%%%%%%%%%%%%%%%%%%%%%%%%%%%%%%%%%%%%%%%%%%%%%%%%%%%%%%%%%%%%%%%%%%%
% GHP derivatives in math mode 
\newcommand{\spindersphere}{{}_s\slashed{\Delta}}
\newcommand{\edth}{\textnormal{\dh}}
\newcommand{\Thorn}{\textnormal{\th}}
\newcommand{\Edth}{\textnormal{\DH}}
\newcommand{\mathTH}{\textnormal{\TH}}
\newcommand{\firstorder}[1]{\dot{#1}}
\newcommand{\secondorder}[1]{\ddot{#1}}
%%%%%%%%%%%%%%%%%%%%%%%%%%%%%%%%%%%%%%%%%%%%%%%%%%%%%%%%%%%%%%%%%%%%%%%%%%%%%%%
% to make overbar (for complex conjugate) easier to read from a distance
\newcommand*\oline[1]{%
   \vbox{%
     \hrule height 1pt%                  % Line above with certain width
     \kern0.25ex%                          % Distance between line and content
     \hbox{%
       \ifmmode#1\else\ensuremath{#1}\fi%  % The content, typeset in dependence of mode
     }
   }
}
%%%%%%%%%%%%%%%%%%%%%%%%%%%%%%%%%%%%%%%%%%%%%%%%%%%%%%%%%%%%%%%%%%%%%%%%%%%%%%%
\usepackage{empheq}
\newcommand*\widefbox[1]{\fbox{\hspace{2em}#1\hspace{2em}}}
\newcommand{\justin}[1]{{\textbf{#1}} }
%%%%%%%%%%%%%%%%%%%%%%%%%%%%%%%%%%%%%%%%%%%%%%%%%%%%%%%%%%%%%%%%%%%%%%%%%%%%%%%
\usepackage{mathalfa,amssymb}
\usepackage{graphicx}
\usepackage{amsmath}
\usepackage{amssymb}
\usepackage{slashed}
\usepackage{graphicx}
\usepackage{setspace}
\usepackage{fullpage}
\usepackage{enumerate}
\usepackage{braket}
\usepackage[hidelinks]{hyperref}

\allowdisplaybreaks

%%%%%%%%%%%%%%%%%%%%%%%%%%%%%%%%%%%%%%%%%%%%%%%%%%%%%%%%%%%%%%%%%%%%%%%%%%%%%%%
\begin{document}

\title{
{Notes on the Newman-Penrose formalism}\\
}
\author{Justin L. Ripley 
   \\ \small{lloydripley[at]gmail[dot]com}
   }
\date{\today}

\maketitle

\abstract{
   These are notes on the Newman-Penrose formalism, which I wrote
   up while working on \cite{Ripley:2020xby}.
   I refer to a Mathematica note several times, which can be accessed at: 

   \url{https://github.com/JLRipley314/2nd-order-teuk-derivations}.

   Nothing in here is new
   (these notes mostly follow Chapter 1 of Chandrasekhar's book
   on black hole perturbation theory \cite{Chandrasekhar_bh_book},
   except for the section on the GHP formalism).
   Please let me know if you find any typos/errors!
}


\tableofcontents

%%%%%%%%%%%%%%%%%%%%%%%%%%%%%%%%%%%%%%%%%%%%%%%%%%%%%%%%%%%%%%%%%%%%%%%%%%%%%%
%%%%%%%%%%%%%%%%%%%%%%%%%%%%%%%%%%%%%%%%%%%%%%%%%%%%%%%%%%%%%%%%%%%%%%%%%%%%%%
\chapter{Setup of basic formalism: Einstein equations and null frames}
\label{chptr:NP_formalism}
%%%%%%%%%%%%%%%%%%%%%%%%%%%%%%%%%%%%%%%%%%%%%%%%%%%%%%%%%%%%%%%%%%%%%%%%%%%%%%
\section{Notation and summary of (sub)Riemannian geometry and the
Einstein equations}
\label{sec:notation}
	We mostly follow the conventions (and order of presentation) of
\cite{Chandrasekhar_bh_book}, except when otherwise noted.
Spacetime indices will be denoted with lower case latin letters. The comma will denote
partial differentiation and the semicolon will denote covariant differentiation,
although I'll also use $\partial_i$ and $\nabla_i$ as well. I will bold font
tensors when I do not give indices.

	The metric signature is $+---$.

	The Ricci identity is
\begin{align}
	\left[\nabla_k,\nabla_l\right]V^j
	=	
	R^j{}_{ikl}V^i
+
	T^n{}_{kl}\nabla_nV^j
	,
\end{align}
	where $T^n{}_{kl}$ is the torsion. We set the torsion to be zero.
	From the Ricci identity, the Riemann tensor is
\begin{align}
	R^j{}_{lnm}
	=
	\Gamma^j_{lm,n}
-	\Gamma^j_{ln,m}
+	\Gamma^j_{kn}\Gamma^k_{lm}
-	\Gamma^j_{km}\Gamma^k_{ln}
	,
\end{align}
	where the $\Gamma^k_{ij}$ are the metric compatible (Christoffel)
connection coefficients. In a coordinate basis we may write 
\begin{align}
	\Gamma^i_{jk}
	=
	\frac{1}{2}g^{il}\left(
		g_{jl,k}
	+	g_{kl,j}
	-	g_{jk,l}
	\right)
	,
\end{align}

The Ricci tensor and Ricci scalar are
\begin{align}
	R_{ij} \equiv R^k{}_{ikj} 
	,
	\qquad
	R \equiv g^{ij}R_{ij}
	.
\end{align}
	The Jacobi identity for the commutator of two derivatives
along with the Ricci identity
gives us the following cyclic identity for the Riemann tensor 
\begin{align}
\label{eq:Riemann_cyclic_identity}
	R^j{}_{lkm}
+	R^j{}_{klm}
+	R^j{}_{mkl}
	=
	0
	.
\end{align} 
	In a coordinate basis, we can write the Bianchi identities as
\begin{align}
\label{eq:Riemann_Bianchi_identity}
	\nabla_rR^j{}_{lpq}
+	\nabla_pR^j{}_{lqr}
+	\nabla_qR^j{}_{lrp}
	=
	0
	.
\end{align}
	The Weyl tensor is
\begin{align}
\label{eq:def_Weyl_tensor}
	C_{ijkl}
	\equiv
	R_{ijkl}
-	\frac{1}{n-2}\left(
		g_{ik}R_{jl}
	-	g_{jk}R_{il}
	-	g_{il}R_{jk}
	+	g_{jl}R_{ik}
	\right)
+	\frac{1}{(n-1)(n-2)}\left(
		g_{ik}g_{jl}
	-	g_{il}g_{jk}
	\right)R
	,
\end{align}
	where $n$ is the dimension of the manifold; we will always work
with $n=4$.
It obeys the same algebraic symmetries as the Riemann tensor, and it is
tracefree on all indices.

	The Einstein equations are
\begin{align}
	G_{ij}\equiv R_{ij}-\frac{1}{2}g_{ij}R
	=
	T_{ij}
	,
\end{align}
	where $G_{ij}$ is the Einstein tensor, and $T_{ij}$ is
the stress-energy tensor. From the Bianchi identities, the Einstein
tensor is divergence free; $\nabla^iG_{ij}=0$.
%%%%%%%%%%%%%%%%%%%%%%%%%%%%%%%%%%%%%%%%%%%%%%%%%%%%%%%%%%%%%%%%%%%%%%%%%%%%%%
\section{Tetrad formalism}
\label{sec:tetrad_formalism}
%%%%%%%%%%%%%%%%%%%%%%%%%%%%%%%%%%%%%%%%%%%%%%%%%%%%%%%%%%%%%%%%%%%%%%%%%%%%%%
\subsection{Basic setup}
	 At every spacetime point we set up a basis of four vectors
\begin{align}
	e^i_{(a)}
	,
\end{align}
	where $a$ is the basis index, and $i$ is the coordinate index.
When we refer to a particular component, e.g. $i=1$, we write 
$e^1_{(a)}$, and if $a=1$ we write $e^i_{(1)}$.
We raise/lower the spacetime indices with the metric/inverse
metric
$g_{ij}$/$g^{ij}$, respectively. 
We raise/lower the basis index with a constant, symmetric
matrix and its inverse $\eta_{(a)(b)}$/$\eta^{(a)(b)}$, respectively.
We call $\eta_{(a)(b)}$ the internal metric. We define $e^{(a)}_i$, etc.
so that 
\begin{align}
	e^i_{(a)} e^{(a)}_j = \delta^i_j
	,
	\qquad
	e^i_{(a)} e^{(b)}_j = \delta^{(b)}_{(a)}
	,
	\qquad
	e^i_{(a)} e_{(b)i} = \eta_{(a)(b)}
	,
	\qquad
	e^{(a)}_ie_{(a)j} = g_{ij}
	,
\end{align}
	where $\delta$ is the Kronecker delta. For any given tensor
we have the notation, e.g. $V^{(a)}=e^{(a)}_iV^i$. 
%%%%%%%%%%%%%%%%%%%%%%%%%%%%%%%%%%%%%%%%%%%%%%%%%%%%%%%%%%%%%%%%%%%%%%%%%%%%%%
\subsection{Intrinsic derivative and Ricci rotation coefficients}
	We define the directional derivatives
\begin{align}
	{\bf e}_{(a)}
	\equiv
	e_{(a)}^i\nabla_i
	.
\end{align}
	Recall for a coordinate basis we have $e^i_{(a)}``="\delta^i_{(a)}$.
The notation for scalars will be 
\begin{align}
	\phi_{,(a)}
	\equiv
	\partial_{(a)}\phi
	\equiv
	e_{(a)}^i\partial_i\phi
	.
\end{align}
	The notation for tensors is
\begin{align}
	\partial_{(b)}V^{(a)}
	\equiv &
	e^j_{(b)}\nabla_j\left(e^{(a)}_iV^i\right)
	\nonumber \\
	= &
	e^j_{(b)} V^i\nabla_je^{(a)}_{i}
+	e^j_{(b)}e^{(a)}_i\nabla_jV^i
	.
\end{align}
	We define the intrinsic derivative
\begin{align}
\label{eq:def_intrinsic_der}
	V^{(a)}{}_{|(b)}
	\equiv
	\nabla_{(b)}V^{(a)}
	\equiv
	e^j_{(b)}e^{(a)}_i\nabla_jV^i
	,	
\end{align}
	and the Ricci rotation coefficients
\begin{align}
\label{eq:def_Ricci_rotation}
	\gamma_{{(c)}{(a)}{(b)}}
	\equiv &
	e_{(c)}^k \left(\nabla_ie_{{(a)}k}\right) e_{(b)}^i
	, \\
\implies \nabla_me_{n(a)}
	= &
	\gamma_{(c)(a)(b)}e^{(c)}_ne^{(b)}_m
	. 
\end{align}
	As $\partial_i\eta_{{(a)}{(b)}}=0$, we have
\begin{align}
	\gamma_{{(c)}{(a)}{(b)}}
	=
-	\gamma_{{(a)}{(c)}{(b)}}
	.
\end{align}
	We next define	
\begin{align}
\label{eq:def_lambda_coef}
	\lambda_{{(a)}{(b)}{(c)}}
	\equiv
	\left(
		\partial_je_{{(b)}i}
	-	\partial_ie_{{(b)}j}
	\right)
	e^i_{(a)}e^j_{(c)}
	.
\end{align}
	With symmetric (e.g. Christoffel)
connections, we may replace the partial derivatives with
covariant derivatives, and we then invert the above relationship
to write the Ricci rotation coefficients in terms
of the $\lambda_{(a)(b)(c)}$:
\begin{align}
	\gamma_{(a)(b)(c)}
	=
	\frac{1}{2}\left(
		\lambda_{(a)(b)(c)}
	+	\lambda_{(c)(a)(b)}
	-	\lambda_{(b)(c)(a)}
	\right)
	.
\end{align} 
%%%%%%%%%%%%%%%%%%%%%%%%%%%%%%%%%%%%%%%%%%%%%%%%%%%%%%%%%%%%%%%%%%%%%%%%%%%%%%
\subsection{Structure constants in terms of Ricci rotation coefficients}
	We mention the structure constants, which are defined to be
\begin{align}
	\left[e^i_{(a)}\nabla_i,e^j_{(b)}\nabla_j\right]
	\equiv
	C^{(c)}{}_{(a)(b)}e^k_{(c)}\nabla_k
	.
\end{align}
	Acting this on a scalar function, we see that
\begin{align}
	C^{(c)}{}_{(a)(b)}
	=
	\gamma^{(c)}{}_{(b)(a)}
-	\gamma^{(c)}{}_{(a)(b)}
	.
\end{align}
%%%%%%%%%%%%%%%%%%%%%%%%%%%%%%%%%%%%%%%%%%%%%%%%%%%%%%%%%%%%%%%%%%%%%%%%%%%%%%
\subsection{Riemann tensor in terms of Ricci rotation coefficients}
	We have
\begin{align}
\label{eq:tetrad_components_Riemann}
	R_{(a)(b)(c)(d)}
	= &
	R_{pqrs}e^p_{(a)} e^q_{(b)} e^r_{(c)} e^s_{(d)}
	\nonumber \\
	= &
	\left(	
		\nabla_s\left(\nabla_re_{q(a)}\right)
	-	\nabla_r\left(\nabla_se_{q(a)}\right)
	\right)
	e^q_{(b)} e^r_{(c)} e^s_{(d)}	
	\nonumber \\
	= &
	\gamma_{(a)(b)(d)|(c)}
-	\gamma_{(a)(b)(c)|(d)}
	\nonumber \\
	&
	+ 	\gamma_{(b)(a)(i)}\gamma_{(c)}{}^{(i)}{}_{(d)}
	- 	\gamma_{(b)(a)(i)}\gamma_{(d)}{}^{(i)}{}_{(c)}
	+	\gamma_{(i)(a)(c)}\gamma_{(b)}{}^{(i)}{}_{(d)}
	-	\gamma_{(i)(a)(d)}\gamma_{(b)}{}^{(i)}{}_{(c)}
	.
\end{align}
%%%%%%%%%%%%%%%%%%%%%%%%%%%%%%%%%%%%%%%%%%%%%%%%%%%%%%%%%%%%%%%%%%%%%%%%%%%%%%
\subsection{Bianchi identity in terms of Ricci Rotation coefficients}
	We write
\begin{align}
\label{eq:tetrad_components_Bianchi_identities}
	R_{(a)(b)[(c)(d)|(f)]}
	= &
	R_{ij[kl;p]}e^i_{(a)} e^j_{(b)} e^k_{(c)} e^l_{(d)} e^p_{(f)}
	\nonumber \\
	= &
	\frac{1}{6}\sum_{[(c)(d)(f)]}\left(
		e^i_{(a)} e^j_{(b)} e^k_{(c)} e^l_{(d)} e^p_{(f)}
		\nabla_rR_{ijkl}
	\right)
	\nonumber \\
	= &
	\frac{1}{6}\sum_{[(c)(d)(f)]}\Big(
		R_{(a)(b)(c)(d),(f)}
	-	\gamma^{(i)}{}_{(a)(f)}R_{(i)(b)(c)(d)}
	-	\gamma^{(i)}{}_{(b)(f)}R_{(a)(i)(c)(d)}
	\nonumber \\
	&\qquad\qquad
	-	\gamma^{(i)}{}_{(c)(f)}R_{(a)(b)(i)(d)}
	-	\gamma^{(i)}{}_{(d)(f)}R_{(a)(b)(c)(i)}
	\Big)
	.
\end{align}
	The symbol $\sum_{[(c)(d)(f)]}$ means
``sum over antisymmetric configurations of $(c)(d)(f)$''. Another way of
writing this is
\begin{align}
	\sum_{[(c)(d)(f)]}=\sum_{(c)(d)(f)}\epsilon_{(c)(d)(f)}
	,
\end{align}
	but do \emph{not} contract indices like in the Einstein summation
convention.
%%%%%%%%%%%%%%%%%%%%%%%%%%%%%%%%%%%%%%%%%%%%%%%%%%%%%%%%%%%%%%%%%%%%%%%%%%%%%%
\subsection{Geodesic equation}
	The geodesic equation is
\begin{align}
	u^i\nabla_iu_j=0
	.
\end{align}
	From the definitions of the Ricci rotation coefficients we can write
this as
\begin{align}
	e^{(j)}_ju^{(i)}
	\left(
		\partial_{(i)}u_{(j)}
	-	\eta^{(n)(m)}\gamma_{(m)(j)(i)}u_{(n)}
	\right)
	=
	0
	.
\end{align}
	The geodesic equation for coordinates may be written as
\begin{subequations}
\begin{align}
	\frac{du_{(j)}}{d\tau}
-	\gamma_{(k)(j)(i)}u^{(i)}u^{(k)}
	= &
	0
	, \\
	\frac{dx^i}{d\tau} - e^i_{(i)}u^{(i)} 
	= &
	0
	.
\end{align}
\end{subequations}
%%%%%%%%%%%%%%%%%%%%%%%%%%%%%%%%%%%%%%%%%%%%%%%%%%%%%%%%%%%%%%%%%%%%%%%%%%%%%%
\section{Newman-Penrose formalism}
\label{sec:np_formalism}
%%%%%%%%%%%%%%%%%%%%%%%%%%%%%%%%%%%%%%%%%%%%%%%%%%%%%%%%%%%%%%%%%%%%%%%%%%%%%%
\subsection{Basic setup}
	The Newman-Penrose formalism is a tetrad formalism with
the following internal metric
\begin{align}
	\eta_{(a)(b)}
	=
	\begin{pmatrix}
	0 & 1 & 0 & 0 \\
	1 & 0 & 0 & 0 \\
	0 & 0 & 0 & -1 \\
	0 & 0 & -1 & 0 	
	\end{pmatrix}
	.
\end{align}
	The four basis vectors have special names
\begin{align}
	l^i
	\equiv
	e^i_{(1)}
	, \qquad
	n^i
	\equiv
	e^i_{(2)}
	, \qquad
	m^i
	\equiv
	e^i_{(3)}
	, \qquad
	\oline{m}^i
	\equiv
	e^i_{(4)}
	.
\end{align}
	Using $\eta_{(a)(b)}$, the we see that
\begin{align}
	e^{i(1)}
	=
	n^i
	, \qquad
	e^{i(2)}
	=
	l^i
	, \qquad
	e^{i(3)}
	=
	-\oline{m}^i
	, \qquad
	e^{i(4)}
	=
	-m^i
	.
\end{align}
	The metric in terms of the NP vectors is 
\begin{align}
	g_{ij}
	=
	n_il_j
+	n_jl_i
-	m_i\oline{m}_j
-	\oline{m}_im_j
	.
\end{align}
	The vectors $\{n^i,l^i\}$ are null and real,
and the vectors 
$\{m^i,\oline{m}^i\}$ are null, complex, and complex conjugates of
each other. We can read off the normalization conditions
from $\eta_{(a)(b)}$:
\begin{align}
	l_im^i=l_i\oline{m}^i=n_im^i=n_i\oline{m}^i
=	l_il^i=n_in^i=m_im^i=\oline{m}_i\oline{m}^i
=	0
	,
\end{align}
	and
\begin{align}
	l_in^i=-m_i\oline{m}^i=1
	.
\end{align}
%%%%%%%%%%%%%%%%%%%%%%%%%%%%%%%%%%%%%%%%%%%%%%%%%%%%%%%%%%%%%%%%%%%%%%%%%%%%%%
\subsection{Definitions:
	directional derivatives and Ricci rotation coefficients}
\label{sec:def_directional_derivatives_Riccci_rotation}
	The directional derivatives have special names
\begin{align}
	D
	\equiv 
	l^i\nabla_i
	, \qquad
	\Delta
	\equiv
	n^i\nabla_i
	, \qquad
	\delta
	\equiv
	m^i\nabla_i
	, \qquad
	\oline{\delta}
	\equiv
	\oline{m}^i\nabla_i
	,
\end{align}
	as do the Ricci rotation coefficients
\begin{align}
\begin{aligned}
\label{eq:def_Ricci_rotation_NP}
	&
	\kappa
	\equiv
	\gamma_{(3)(1)(1)}
	, &
	\sigma
	\equiv
	\gamma_{(3)(1)(3)}
	, \\
	&
	\lambda
	\equiv
	\gamma_{(2)(4)(4)}
	, &
	\nu
	\equiv
	\gamma_{(2)(4)(2)}
	, \\
	&
	\rho
	\equiv
	\gamma_{(3)(1)(4)}
	, &
	\mu
	\equiv
	\gamma_{(2)(4)(3)}
	, \\
	&
	\tau
	\equiv
	\gamma_{(3)(1)(2)}
	, &
	\pi
	\equiv
	\gamma_{(2)(4)(1)}
	, \\
	&
	\epsilon
	\equiv
	\frac{1}{2}\left(
		\gamma_{(2)(1)(1)}
	+	\gamma_{(3)(4)(1)}
	\right)
	, &
	\gamma
	\equiv
	\frac{1}{2}\left(
		\gamma_{(2)(1)(2)}
	+	\gamma_{(3)(4)(2)}
	\right)
	, \\
	&
	\alpha
	\equiv
	\frac{1}{2}\left(
		\gamma_{(2)(1)(4)}
	+	\gamma_{(3)(4)(4)}
	\right)
	, &
	\beta
	\equiv
	\frac{1}{2}\left(
		\gamma_{(2)(1)(3)}
	+	\gamma_{(3)(4)(3)}
	\right)
	.
\end{aligned}
\end{align}
	These definitions, combined with the antisymmetry of
$\gamma_{(a)(b)(c)}$ in its first two indices and the $3\leftrightarrow4$
rule with complex conjugate (
	We can find the complex conjugates of these by 
setting $3\to4$ and $4\to3$, as those are the only complex quantities) 
, allows us to express all 24 Ricci rotation
coefficients in terms of the above 12 complex Newman-Penrose scalars.
For example we can invert the last four relationships to get
\begin{align}
\label{eq:definite_ricci_rot_NP_scalars}
\begin{aligned}
	&
	\gamma_{(2)(1)(1)}
	=
	\epsilon + \oline{\epsilon}
	, &
	\gamma_{(3)(4)(1)}
	=
	\epsilon - \oline{\epsilon}
	, \\
	&
	\gamma_{(2)(1)(2)}
	=
	\gamma + \oline{\gamma}
	, &
	\gamma_{(3)(4)(2)}
	=
	\gamma - \oline{\gamma}
	, \\
	&
	\gamma_{(2)(1)(4)}
	=
	\alpha + \oline{\beta}
	, &
	\gamma_{(3)(4)(4)}
	=
	\alpha - \oline{\beta}
	.
\end{aligned}
\end{align} 
%%%%%%%%%%%%%%%%%%%%%%%%%%%%%%%%%%%%%%%%%%%%%%%%%%%%%%%%%%%%%%%%%%%%%%%%%%%%%%
\subsection{Definitions: components of Weyl and Ricci tensors}
\label{sec:def_Weyl_Ricci_tensors}
	We want to find the tetrad components of the curvature tensors. The
Riemann tensor can be decomposed into the Weyl and Ricci tensors, and it
is the components of those tensors that have special names. In four dimensions
both tensors have 10 components, which are written as five complex scalars.
First we list the Weyl scalars 
\begin{align}
\begin{aligned}
	\Psi_0
	\equiv
	-C_{pqrs}l^pm^ql^rm^s
	, \\
	\Psi_1
	\equiv
	-C_{pqrs}l^pn^ql^rm^s
	, \\
	\Psi_2
	\equiv
	-C_{pqrs}l^pm^q\oline{m}^rn^s
	, \\
	\Psi_3
	\equiv
	-C_{pqrs}l^pn^q\oline{m}^rn^s
	, \\
	\Psi_4
	\equiv
	-C_{pqrs}n^p\oline{m}^qn^r\oline{m}^s
	.
\end{aligned}
\end{align} 
	Chandrasekhar \cite{Chandrasekhar_bh_book}
describes a combinatorial procedure
(which I implement in my Mathematica note) that allows one to
write all the nonzero tetrad contractions of the Weyl
tensor in terms of the above five complex Weyl scalars.

	There are four real and three complex
Ricci terms. {\bf NOTE:}
Newman and Penrose \cite{Newman_Penrose_paper} use
$R_{ij}=R^k{}_{ijk}$, while we use the conventions Chandrasekhar
\cite{Chandrasekhar_bh_book}:$R_{ij}=R^k{}_{ikj}$.
Thus, to make our NP equations
look the same as in NP's original paper, we define the below
terms with a global change of sign from what you
will find in the original NP paper.
Note also that there appears to be a global
sign flip typo in Chandrasekhar's formulas that include the Ricci terms:
he should define them as we do below to get the NP equations he later derives
\begin{align}
\begin{aligned} 
	&
	\Phi_{00}
	\equiv
	\frac{1}{2}R_{pq}l^pl^q
	, &
	\Phi_{22}
	\equiv
	\frac{1}{2}R_{pq}n^pn^q
	, \\
	&
	\Phi_{02}
	\equiv
	\frac{1}{2}R_{pq}m^pm^q
	, &
	\Phi_{20}
	\equiv
	\frac{1}{2}R_{pq}\oline{m}^p\oline{m}^q
	, \\
	&
	\Phi_{11}
	\equiv
	\frac{1}{4}R_{pq}\left(
		l^pn^q
	+	m^p\oline{m}^q
	\right)
	, &
	\Phi_{01}
	\equiv
	\frac{1}{2}R_{pq}l^pm^q
	, \\
	&
	\Phi_{10}
	\equiv
	\frac{1}{2}R_{pq}l^p\oline{m}^q
	, &
	\Phi_{12}
	\equiv
	\frac{1}{2}R_{pq}n^pm^q
	, \\
	&
	\Phi_{21}
	\equiv
	\frac{1}{2}R_{pq}n^p\oline{m}^q
	, & 
	\Lambda
	\equiv
	-\frac{1}{24}R
	.
\end{aligned}
\end{align} 
	As the Ricci tensor is symmetric it is straightforward to write
the rest of the Ricci tensor components in terms of the above NP scalars. 
%%%%%%%%%%%%%%%%%%%%%%%%%%%%%%%%%%%%%%%%%%%%%%%%%%%%%%%%%%%%%%%%%%%%%%%%%%%%%%
\subsection{Derived relations: commutation relations}
\label{sec:commutation_relations}
	We obtain relations for commuting the directional derivatives
through the relation
\begin{align}
	\left[e^i_{(a)}\nabla_i,e^j_{(b)}\nabla_j\right]
	=
	C^{(c)}{}_{(a)(b)}e_{(c)}^k\nabla_k
	=
	\left(
		\gamma_{(c)(b)(a)}
	-	\gamma_{(c)(a)(b)}
	\right)
	e^{k(c)}\nabla_k
	.
\end{align}
	An exercise in applying the definitions gives us	
\begin{subequations}
\label{eq:commutation_relations}
\begin{align}
\label{eq:commutation_D_Delta}
	\left[D,\Delta\right]
	=
-	\left(\gamma+\oline{\gamma}\right)D
-	\left(\epsilon+\oline{\epsilon}\right)\Delta
+	\left(\pi+\oline{\tau}\right)\delta
+	\left(\oline{\pi}+\tau\right)\oline{\delta}
	, \\
\label{eq:commutation_D_delta}
	\left[D,\delta\right]
	=
	\left(\oline{\pi} - \beta - \oline{\alpha}
	\right)D
-	\kappa\Delta
+	\left(\epsilon - \oline{\epsilon}+\oline{\rho}\right)
	\delta
+	\sigma\oline{\delta}
	, \\
\label{eq:commutation_Delta_delta}
	\left[\Delta,\delta\right]
	=
	\oline{\nu}D
+	\left(-\tau+\oline{\alpha}+\beta\right)\Delta
+	\left(\gamma-\oline{\gamma}-\mu\right)\delta
-	\oline{\lambda}\oline{\delta}
	, \\
\label{eq:commutation_delta_deltacc}
	\left[\delta,\oline{\delta}\right]
	=
	\left(\mu-\oline{\mu}\right)D
+	\left(\rho-\oline{\rho}\right)\Delta
+	\left(-\alpha+\oline{\beta}\right)\delta
+	\left(\oline{\alpha}-\beta\right)\oline{\delta}
	.
\end{align}
\end{subequations}
	These are (1.303)-(1.306) in \cite{Chandrasekhar_bh_book}.
%%%%%%%%%%%%%%%%%%%%%%%%%%%%%%%%%%%%%%%%%%%%%%%%%%%%%%%%%%%%%%%%%%%%%%%%%%%%%%
\subsection{Derived relations: components of Riemann tensor
(``Ricci identities'')}
\label{sec:components_Riemann tensor}
	Using Eq.~\eqref{eq:tetrad_components_Riemann} for the tetrad
projections of the Riemann tensor, and the definitions
\eqref{eq:def_Ricci_rotation_NP} for the Ricci rotation coefficients,
we can write down the independent components of the Riemann tensor.	
We can equate these expression to the components of the Riemann tensor
written in terms of the NP scalars $\Psi_0$, $\Phi_{00}$, etc, using 
Eq.~\eqref{eq:def_Weyl_tensor}.

	We implement this in Mathematica and do not include the output here.
See also \cite{Chandrasekhar_bh_book}, Chptr 1, Eq.(310), and
\cite{Newman_Penrose_paper} (note though that
NP use opposite signs for Ricci rotation
coefficients than Chandrasekhar). These essentially give us transport
equations for the Ricci rotation coefficients $\alpha$, etc. 

By taking linear combinations of these
equations (and complex conjugates of them) we can eliminate
some of the scalars $\Phi_{00}$, etc. and get the so-call
``eliminant relations'' (\cite{Chandrasekhar_bh_book},
Chptr 1, Eq.(310)).
The Ricci identities as derived in the Mathematica notebook are
(this numbering follows \cite{Chandrasekhar_bh_book}, Eq. 310, and the
equations listed here should exactly match those) 
\begin{subequations}
\label{eq:ricci_identities_np}
\begin{align}
	\kappa\left(-3\alpha - \oline{\beta} + \pi\right) + \rho\left(\epsilon + \oline{\epsilon} + \rho\right) + \sigma\oline{\sigma} - \oline{\kappa}\tau + \Phi_{00} 
	- D\left(\rho\right) + \oline{\delta}\left(\kappa\right)
	= &
	0
	, \\
	\left(3\epsilon - \oline{\epsilon} + \rho + \oline{\rho}\right)\sigma - \kappa\left(\oline{\alpha} + 3\beta - \oline{\pi} + \tau\right) + \Psi_{0} 
	- D\left(\sigma\right) + \delta\left(\kappa\right)
	= &
	0
	, \\
	-\left(\left(3\gamma + \oline{\gamma}\right)\kappa\right) + \oline{\pi}\rho + \left(\epsilon - \oline{\epsilon} + \rho\right)\tau + \sigma\left(\pi + \oline{\tau}\right) + \Phi_{01} + \Psi_{1} 
	- D\left(\tau\right) + \Delta\left(\kappa\right)
	= &
	0
	, \\
	-\left(\oline{\beta}\epsilon\right) - \gamma\oline{\kappa} - \kappa\lambda + \pi\left(\epsilon + \rho\right) + \alpha\left(-2\epsilon + \oline{\epsilon} + \rho\right) + \beta\oline{\sigma} + \Phi_{10} 
	- D\left(\alpha\right) + \oline{\delta}\left(\epsilon\right)
	= &
	0
	, \\
	\kappa\left(\gamma + \mu\right) + \epsilon\left(\oline{\alpha} - \oline{\pi}\right) + \beta\left(\oline{\epsilon} - \oline{\rho}\right) - \left(\alpha + \pi\right)\sigma - \Psi_{1} 
	+ D\left(\beta\right) - \delta\left(\epsilon\right)
	= &
	0
	, \\
	2\gamma\epsilon + \oline{\gamma}\epsilon + \gamma\oline{\epsilon} + \Lambda + \kappa\nu - \pi\left(\beta + \tau\right) - \alpha\left(\oline{\pi} + \tau\right) - \beta\oline{\tau} - \Phi_{11} - \Psi_{2} 
	+ D\left(\gamma\right) - \Delta\left(\epsilon\right)
	= & 
	0
	, \\
	3\epsilon\lambda + \oline{\kappa}\nu - \pi\left(\alpha - \oline{\beta} + \pi\right) - \lambda\left(\oline{\epsilon} + \rho\right) - \mu\oline{\sigma} - \Phi_{20} 
	+ D\left(\lambda\right) - \oline{\delta}\left(\pi\right)
	= &
	0
	, \\
	-2\Lambda + \kappa\nu + \oline{\alpha}\pi - \pi\left(\beta + \oline{\pi}\right) + \mu\left(\epsilon + \oline{\epsilon} - \oline{\rho}\right) - \lambda\sigma - \Psi_{2} 
	+ D\left(\mu\right) - \delta\left(\pi\right)
	= &
	0
	, \\
	\left(3\epsilon + \oline{\epsilon}\right)\nu - \gamma\pi + \oline{\gamma}\pi - \lambda\left(\oline{\pi} + \tau\right) - \mu\left(\pi + \oline{\tau}\right) - \Phi_{21} - \Psi_{3} 
	+ D\left(\nu\right) - \Delta\left(\pi\right)
	= &
	0
	, \\
	\lambda\left(3\gamma - \oline{\gamma} + \mu + \oline{\mu}\right) - \nu\left(3\alpha + \oline{\beta} + \pi - \oline{\tau}\right) + \Psi_{4} 
	+ \Delta\left(\lambda\right) - \oline{\delta}\left(\nu\right)
	= &
	0
	, \\
	\kappa\left(-\mu + \oline{\mu}\right) - \left(\oline{\alpha} + \beta\right)\rho + 3\alpha\sigma - \oline{\beta}\sigma + \left(-\rho + \oline{\rho}\right)\tau - \Phi_{01} + \Psi_{1} 
	+ \delta\left(\rho\right) - \oline{\delta}\left(\sigma\right)
	= &
	0
	, \\
	-\left(\alpha\oline{\alpha}\right) + 2\alpha\beta - \beta\oline{\beta} - \Lambda + \epsilon\left(-\mu + \oline{\mu}\right) - \left(\gamma + \mu\right)\rho + \gamma\oline{\rho} + \lambda\sigma - \Phi_{11} + \Psi_{2} 
	+ \delta\left(\alpha\right) - \oline{\delta}\left(\beta\right)
	= &
	0
	, \\
	-\left(\oline{\alpha}\lambda\right) + 3\beta\lambda - \left(\alpha + \oline{\beta}\right)\mu + \left(-\mu + \oline{\mu}\right)\pi - \nu\rho + \nu\oline{\rho} - \Phi_{21} + \Psi_{3} 
	+ \delta\left(\lambda\right) - \oline{\delta}\left(\mu\right)
	= &
	0
	, \\
	-\left(\lambda\oline{\lambda}\right) - \mu\left(\gamma + \oline{\gamma} + \mu\right) + \oline{\nu}\pi + \nu\left(\oline{\alpha} + 3\beta - \tau\right) - \Phi_{22} 
	+ \delta\left(\nu\right) - \Delta\left(\mu\right)
	= &
	0
	, \\
	\oline{\alpha}\gamma + 2\beta\gamma - \beta\oline{\gamma} - \alpha\oline{\lambda} - \beta\mu + \epsilon\oline{\nu} + \nu\sigma - \left(\gamma + \mu\right)\tau - \Phi_{12} 
	+ \delta\left(\gamma\right) - \Delta\left(\beta\right)
	= &
	0
	, \\
	\kappa\oline{\nu} - \oline{\lambda}\rho + 3\gamma\sigma - \left(\oline{\gamma} + \mu\right)\sigma + \oline{\alpha}\tau - \tau\left(\beta + \tau\right) - \Phi_{02} 
	+ \delta\left(\tau\right) - \Delta\left(\sigma\right)
	= &
	0
	, \\ 
	2\Lambda - \kappa\nu - \left(\gamma + \oline{\gamma} - \oline{\mu}\right)\rho + \lambda\sigma + \tau\left(\alpha - \oline{\beta} + \oline{\tau}\right) + \Psi_{2} 
	+ \Delta\left(\rho\right) - \oline{\delta}\left(\tau\right)
	= &
	0
	, \\
	\alpha\left(\oline{\gamma} - \oline{\mu}\right) + \nu\left(\epsilon + \rho\right) - \lambda\left(\beta + \tau\right) + \gamma\left(\oline{\beta} - \oline{\tau}\right) - \Psi_{3} 
	- \Delta\left(\alpha\right) + \oline{\delta}\left(\gamma\right)
	= &
	0
	.
\end{align}
\end{subequations} 
%%%%%%%%%%%%%%%%%%%%%%%%%%%%%%%%%%%%%%%%%%%%%%%%%%%%%%%%%%%%%%%%%%%%%%%%%%%%%%
\subsection{Derived: Weyl scalars in terms of the Ricci rotation coefficients}
\label{sec:Weyl_scalars_Ricci_rotation_coefficients}
	We can think of the results in
Sec.~\ref{sec:components_Riemann tensor} as providing definitions for
the Weyl scalars in terms of derivatives and polynomial combinations of
the Ricci rotation coefficients. We have 
\begin{subequations}
\label{eq:Weyl_scalars}
\begin{align}
	\Psi_0
	= &
	\oline{\alpha}\kappa + 3\beta\kappa - \kappa\oline{\pi} - 3\epsilon\sigma + \oline{\epsilon}\sigma - \rho\sigma - \oline{\rho}\sigma + \kappa\tau + D\left(\sigma\right) - \delta\left(\kappa\right)
	, \\
	\Psi_1
	= &
	\oline{\alpha}\epsilon + \beta\oline{\epsilon} + \gamma\kappa + \kappa\mu - \epsilon\oline{\pi} - \beta\oline{\rho} - \alpha\sigma - \pi\sigma + D\left(\beta\right) - \delta\left(\epsilon\right)
	, \\
	\Psi_2
	= &
	-2\Lambda + \epsilon\mu + \oline{\epsilon}\mu + \kappa\nu + \oline{\alpha}\pi - \beta\pi - \pi\oline{\pi} - \mu\oline{\rho} - \lambda\sigma + D\left(\mu\right) - \delta\left(\pi\right)
	, \\
	\Psi_3
	= &
	\oline{\beta}\gamma + \alpha\oline{\gamma} - \beta\lambda - \alpha\oline{\mu} + \epsilon\nu + \nu\rho - \lambda\tau - \gamma\oline{\tau} - \Delta\left(\alpha\right) + \oline{\delta}\left(\gamma\right)
	, \\
	\Psi_4
	= &
	-3\gamma\lambda + \oline{\gamma}\lambda - \lambda\mu - \lambda\oline{\mu} + 3\alpha\nu + \oline{\beta}\nu + \nu\pi - \nu\oline{\tau} - \Delta\left(\lambda\right) + \oline{\delta}\left(\nu\right)
	.
\end{align} 
\end{subequations}
	These equations are highlighted in the Mathematica note in the
computation of the Ricci identities. 
%%%%%%%%%%%%%%%%%%%%%%%%%%%%%%%%%%%%%%%%%%%%%%%%%%%%%%%%%%%%%%%%%%%%%%%%%%%%%%
\subsection{Derived: Components of the Bianchi identities}
\label{sec:components_Bianchi_identities}
	Using Eq.~\eqref{eq:tetrad_components_Bianchi_identities} and
the definitions for the NP scalars for the Ricci rotation coefficients,
we can derive long expressions for the Bianchi identities.
We can obtain shorter expressions by writing 
the Riemann tensor in terms of the Weyl and Ricci scalars using 
Eq.~\eqref{eq:def_Weyl_tensor}.
The Bianchi identities then become first order differential equations for those
scalars quantities. These are again derived in the Mathematica note. 
The Bianchi identities written in this form are
(c.f. Eq.~(321) in \cite{Chandrasekhar_bh_book})
\begin{subequations}
\begin{align}
	4\alpha\Psi_{0} - \pi\Psi_{0} - 2\left(\epsilon + 2\rho\right)\Psi_{1} + 3\kappa\Psi_{2} + D\left(\Psi_{1}\right) - \oline{\delta}\left(\Psi_{0}\right)
	= &
	\mathcal{R}_a
	, \\
	-\left(\lambda\Psi_{0}\right) - 2\alpha\Psi_{1} + 2\pi\Psi_{1} + 3\rho\Psi_{2} - 2\kappa\Psi_{3} - D\left(\Psi_{2}\right) + \oline{\delta}\left(\Psi_{1}\right)
	= &
	\mathcal{R}_b
	, \\
	2\lambda\Psi_{1} - 3\pi\Psi_{2} + 2\epsilon\Psi_{3} - 2\rho\Psi_{3} + \kappa\Psi_{4} + D\left(\Psi_{3}\right) - \oline{\delta}\left(\Psi_{2}\right)
	= &
	\mathcal{R}_c
	, \\
	-3\lambda\Psi_{2} + 2\alpha\Psi_{3} + 4\pi\Psi_{3} - 4\epsilon\Psi_{4} + \rho\Psi_{4} - D\left(\Psi_{4}\right) + \oline{\delta}\left(\Psi_{3}\right)
	= &
	\mathcal{R}_d
	, \\
	4\gamma\Psi_{0} - \mu\Psi_{0} - 2\left(\beta + 2\tau\right)\Psi_{1} + 3\sigma\Psi_{2} + \delta\left(\Psi_{1}\right) - \Delta\left(\Psi_{0}\right)
	= &
	\mathcal{R}_e
	, \\
	\nu\Psi_{0} + 2\gamma\Psi_{1} - 2\mu\Psi_{1} - 3\tau\Psi_{2} + 2\sigma\Psi_{3} + \delta\left(\Psi_{2}\right) - \Delta\left(\Psi_{1}\right)
	= &
	\mathcal{R}_f
	, \\
	2\nu\Psi_{1} - 3\mu\Psi_{2} + 2\beta\Psi_{3} - 2\tau\Psi_{3} + \sigma\Psi_{4} + \delta\left(\Psi_{3}\right) - \Delta\left(\Psi_{2}\right)
	= &
	\mathcal{R}_g
	, \\
	3\nu\Psi_{2} - 2\gamma\Psi_{3} - 4\mu\Psi_{3} + 4\beta\Psi_{4} - \tau\Psi_{4} + \delta\left(\Psi_{4}\right) - \Delta\left(\Psi_{3}\right)
	= &
	\mathcal{R}_h
	,
\end{align}
\end{subequations} 
	where the $\mathcal{R}$ are Ricci scalar terms
\begin{subequations}
\begin{align}
	\mathcal{R}_a
	\equiv &
	\left(2\left(\oline{\alpha} + \beta\right) - \oline{\pi}\right)\Phi_{00} - 2\left(\epsilon + \oline{\rho}\right)\Phi_{01} + \oline{\kappa}\Phi_{02} - 2\sigma\Phi_{10} + 2\kappa\Phi_{11} 
	\nonumber \\ &
	+ D\left(\Phi_{01}\right) - \delta\left(\Phi_{00}\right)
	, \\
	\mathcal{R}_b
	\equiv &
	\left(-2\left(\gamma + \oline{\gamma}\right) + \oline{\mu}\right)\Phi_{00} + 2\left(\alpha + \oline{\tau}\right)\Phi_{01} - \oline{\sigma}\Phi_{02} + 2\tau\Phi_{10} - 2\rho\Phi_{11} 
	\nonumber \\ &
	+ 2D\left(\Lambda\right) + \Delta\left(\Phi_{00}\right) - \oline{\delta}\left(\Phi_{01}\right)
	, \\
	\mathcal{R}_c
	\equiv &
	2\mu\Phi_{10} - 2\pi\Phi_{11} + 2\oline{\alpha}\Phi_{20} - \left(2\beta + \oline{\pi}\right)\Phi_{20} + 2\epsilon\Phi_{21} - 2\oline{\rho}\Phi_{21} + \oline{\kappa}\Phi_{22} 
	\nonumber \\ &
	+ D\left(\Phi_{21}\right) - \delta\left(\Phi_{20}\right) + 2\oline{\delta}\left(\Lambda\right)
	, \\
	\mathcal{R}_d
	\equiv &
	-2\nu\Phi_{10} + 2\lambda\Phi_{11} + \left(2\gamma - 2\oline{\gamma} + \oline{\mu}\right)\Phi_{20} - 2\alpha\Phi_{21} + 2\oline{\tau}\Phi_{21} - \oline{\sigma}\Phi_{22} 
	\nonumber \\ &
	+ \Delta\left(\Phi_{20}\right) - \oline{\delta}\left(\Phi_{21}\right)
	, \\
	\mathcal{R}_e
	\equiv &
	\oline{\lambda}\Phi_{00} + 2\left(\beta - \oline{\pi}\right)\Phi_{01} - \left(2\epsilon - 2\oline{\epsilon} + \oline{\rho}\right)\Phi_{02} - 2\sigma\Phi_{11} + 2\kappa\Phi_{12} 
	\nonumber \\ &
	+ D\left(\Phi_{02}\right) - \delta\left(\Phi_{01}\right)
	, \\
	\mathcal{R}_f
	\equiv &
	\oline{\nu}\Phi_{00} + 2\left(\gamma - \oline{\mu}\right)\Phi_{01} - \left(2\alpha - 2\oline{\beta} + \oline{\tau}\right)\Phi_{02} - 2\tau\Phi_{11} + 2\rho\Phi_{12} 
	\nonumber \\ &
	- 2\delta\left(\Lambda\right) - \Delta\left(\Phi_{01}\right) + \oline{\delta}\left(\Phi_{02}\right)
	, \\
	\mathcal{R}_g
	\equiv &
	2\mu\Phi_{11} - 2\pi\Phi_{12} + \oline{\lambda}\Phi_{20} - 2\left(\beta + \oline{\pi}\right)\Phi_{21} + \left(2\left(\epsilon + \oline{\epsilon}\right) - \oline{\rho}\right)\Phi_{22} 
	\nonumber \\ &
	+ D\left(\Phi_{22}\right) - \delta\left(\Phi_{21}\right) + 2\Delta\left(\Lambda\right)
	, \\
	\mathcal{R}_h
	\equiv &
	2\nu\Phi_{11} - 2\lambda\Phi_{12} + \oline{\nu}\Phi_{20} - 2\left(\gamma + \oline{\mu}\right)\Phi_{21} + \left(2\left(\alpha + \oline{\beta}\right) - \oline{\tau}\right)\Phi_{22} 
	\nonumber \\ & 
	- \Delta\left(\Phi_{21}\right) + \oline{\delta}\left(\Phi_{22}\right)
	.
\end{align}
\end{subequations} 
	The Ricci scalar terms are
all zero for vacuum spacetime solutions to the Einstein equations.
%%%%%%%%%%%%%%%%%%%%%%%%%%%%%%%%%%%%%%%%%%%%%%%%%%%%%%%%%%%%%%%%%%%%%%%%%%%%%%
\subsection{Transformations of null frame}
	There are $4\times4=16$ degrees of freedom with the choice
of the four null vectors $\{l^i,n^i,m^i,\oline{m}^i\}$. The 10 conditions
\begin{align}
\begin{aligned}
	l_il^i=n_in^i=m_im^i=\oline{m}_i\oline{m}^i
	=l_im^i=l_i\oline{m}^i=n_im^i=n_i\oline{m}^i
	= 0
	, \\
	l_in^i=-m_i\oline{m}^i=1
	,
\end{aligned}
\end{align} 
	leave us with 6 transformations, which we can identify as the
Lorentz transformations (``rotations'')
at any given tangent space. 
Note that picking a null frame is independent of picking a set
of coordinates/gauge for the manifold. When working in the NP formalism
we have 6 choices of null frame transformations and 4 choices of gauge
transformations.

Following
\cite{Chandrasekhar_bh_book}, we consider three classes of transformations
that preserve the above constraints

\begin{enumerate}[I]
\item rotations which leave $l$ unchanged 
\begin{align}
\begin{aligned}
	l^i & \to l^i , \\
	n^i & \to n^i + \oline{a} m^i + a \oline{m}^i + a \oline{a} l^i  , \\
	m^i & \to m^i + a l^i
	.
\end{aligned}
\end{align}
\item rotations which leave $n$ unchanged 
\begin{align}
\begin{aligned}
	n^i & \to n^i , \\
	l^i & \to l^i + \oline{b} m^i + b \oline{m}^i + b \oline{b} n^i  , \\
	m^i & \to m^i + b n^i
	.
\end{aligned}
\end{align}
\item rotations which leave directions of $l$ and $n$ unchanged and rotate
the vectors $m$ and $\oline{m}$ 
\begin{align}
\begin{aligned}
	l^i & \to A^{-1} n^i , \\
	n^i & \to A n^i , \\
	m^i & \to e^{i\theta} m^i
	.
\end{aligned}
\end{align}
\end{enumerate}

	These three classes form a basis for the 6 Lorentz transformations:
two complex numbers $a$ and $b$, and two real numbers $A$ and $\theta$. We
can determine how the Weyl scalars, etc. transform under the above
transformations using their definitions. We list how the Weyl scalars
transform as this will be important for the Petrov classification of
spacetimes. In deriving these relations we used the fact that, due
to the symmetries of the Weyl tensor, many components are zero (this
can be found in the Mathematica note).

	Class $I$:
\begin{align}
\begin{aligned}
\label{eq:null_frame_transformations_class_one_Weyl_scalars}
	\Psi_0 & \to \Psi_0 , \\
	\Psi_1 & \to \Psi_1 + \oline{a} \Psi_0 , \\
	\Psi_2 & \to \Psi_2 + 2\oline{a}\Psi_1 + (\oline{a})^2\Psi_0 , \\ 
	\Psi_3 & \to \Psi_3 + 2\oline{a}\Psi_2 + 3(\oline{a})^2\Psi_1 + (\oline{a})^3\Psi_0 , \\
	\Psi_4 & \to \Psi_4 + 4\oline{a}\Psi_3 + 6(\oline{a})^2\Psi_2 + 4(\oline{a})^3\Psi_1 + (\oline{a})^4\Psi_0 
	.
\end{aligned}
\end{align}

	Class $II$:
\begin{align}
\begin{aligned}
\label{eq:null_frame_transformations_class_two_Weyl_scalars}
	\Psi_0 & \to \Psi_0 + 4b\Psi_1 + 6b^2\Psi_2 + 4b^3\Psi_3 + b^4\Psi_4 , \\
	\Psi_1 & \to \Psi_1 + 3b\Psi_2 + 3b^2\Psi_3 + b^3\Psi_4 , \\
	\Psi_2 & \to \Psi_2 + 2b\Psi_3 + b^2\Psi_4 , \\ 
	\Psi_3 & \to \Psi_3 + b\Psi_4 , \\
	\Psi_4 & \to \Psi_4
	.
\end{aligned}
\end{align}

	Class $III$:
\begin{align}
\begin{aligned}
\label{eq:null_frame_transformations_class_three_Weyl_scalars}
	\Psi_0 & \to A^{-2}e^{2i\theta}\Psi_0 , \\
	\Psi_1 & \to A^{-1}e^{i\theta}\Psi_1 , \\
	\Psi_2 & \to \Psi_2 , \\ 
	\Psi_3 & \to A e^{-i\theta}\Psi_3 , \\
	\Psi_4 & \to A^2 e^{-2i\theta}\Psi_4
	.
\end{aligned}
\end{align}
%%%%%%%%%%%%%%%%%%%%%%%%%%%%%%%%%%%%%%%%%%%%%%%%%%%%%%%%%%%%%%%%%%%%%%%%%%%%%%
\subsection{Invariance under swapping of null frame}
\label{eq:invariant_NP_eqns_nlmm_switch}
	The NP equations taken as a set
are invariant under \cite{GHP_paper}
\begin{align}
	l^i\leftrightarrow n^i\qquad m^i\leftrightarrow\oline{m}^i
	.
\end{align}
	Under this transformation we have
\begin{align}
\label{eq:nlm_switch_ders}
	& D\leftrightarrow
	\Delta
	, 
	& \delta\leftrightarrow
	\oline{\delta}
	,
\end{align}
	and
\begin{align}
\begin{aligned}
\label{eq:nlm_switch_Rici_rot}
	& \kappa
	\leftrightarrow
-	\nu
	,
	& \sigma
	\leftrightarrow
-	\lambda
	, \\
	& \rho
	\leftrightarrow
-	\mu	
	, 
	& \tau
	\leftrightarrow 
-	\pi	
	, \\
	& \epsilon
	\leftrightarrow
-	\gamma
	,
	& \alpha
	\leftrightarrow
-	\beta
	,
\end{aligned}
\end{align}
	and
\begin{align}
\begin{aligned}
\label{eq:nlm_switch_Psi}
	\Psi_0
	\leftrightarrow
	\Psi_4
	, 
	\qquad \Psi_1
	\leftrightarrow
	\Psi_3
	,
	\qquad \Psi_2
	\leftrightarrow
	\Psi_2
	.
\end{aligned}
\end{align}
%%%%%%%%%%%%%%%%%%%%%%%%%%%%%%%%%%%%%%%%%%%%%%%%%%%%%%%%%%%%%%%%%%%%%%%%%%%%%%
\section{Spin and boost weight of the NP scalars}
	GHP \cite{GHP_paper} first pointed out that we can classify
the NP scalars (or algebraic combinations of them) into quantities of
definite spin (and `boost') weight. As this is important for how we
expand the NP scalar in terms of spin weighted spherical harmonics,
we go over some of the main parts of their formalism here. 

	We classify the spin and boost weight of a NP scalar by considering
how it transforms under the following complex rotation
\begin{subequations}
\begin{align}
	l^i\to&\xi\oline{\xi}l^i
	,\\
	n^i\to&\xi^{-1}\oline{\xi}^{-1}l^i
	,\\
	m^i\to&\xi\oline{\xi}^{-1}m^i
	,\\
	\oline{m}^i\to&\xi^{-1}\oline{\xi}\oline{m}^i
	,
\end{align}
\end{subequations}
	where $\xi$ is a complex number.
A scalar that transforms as
\begin{align}
	\eta\to\lambda^p\oline{\lambda}^q\eta,
\end{align}
	has weight $\{p,q\}$, spin weight $(p-q)/2$
and boost weight $(p+q)/2$. From this definition, we see that the
NP scalar of definite weight are 

\begin{center}
\begin{tabular}{ c c c c }
\hline
 NP scalar & weight &  spin weight &  boost weight \\ 
\hline
$\Psi_0$ 	& $\{ 4, 0\}$ & $ 2$  & $ 2$ \\
$\Psi_1$ 	& $\{ 2, 0\}$ & $ 1$  & $ 1$ \\
$\Psi_2$ 	& $\{ 0, 0\}$ & $ 0$  & $ 0$ \\
$\Psi_3$ 	& $\{-2, 0\}$ & $-1$ & $-1$ \\
$\Psi_4$ 	& $\{-4, 0\}$ & $-2$ & $-2$ \\ 
$\sigma$ 	& $\{ 3,-1\}$ & $ 2$ & $ 1$ \\ 
$\kappa$ 	& $\{ 3, 1\}$ & $ 1$ & $ 2$ \\ 
$\tau$ 		& $\{ 1,-1\}$ & $ 1$ & $ 0$ \\ 
$\rho$ 		& $\{ 1, 1\}$ & $ 0$ & $ 1$ \\ 
$\mu$ 		& $\{-1,-1\}$ & $ 0$ & $-1$ \\ 
$\pi$ 		& $\{-1, 1\}$ & $-1$ & $ 0$ \\ 
$\nu$ 		& $\{-3,-1\}$ & $-1$ & $-2$ \\ 
$\lambda$ 	& $\{-3, 1\}$ & $-2$ & $-1$ \\ 
\end{tabular}
\end{center}

	The NP scalars of indefinite weight are
$\{\alpha,\beta,\gamma,\epsilon\}$, as $\nabla_i\xi\neq0$, so
if we have e.g. $n^i\nabla_kl_i$, then the Ricci rotation coefficient
would have terms like $\nabla_k\left(\xi\oline{\xi}\right)$.
Note as well that the derivative
operators $\{D,\Delta,\delta,\oline{\delta}\}$ have definite spin weight
when considered by themselves, but we cannot think of them as having
definite spin weight when they act of a scalar. To make them operators
of definite spin weight, acting on a scalar of weight $\{p,q\}$ we define
\begin{subequations}
\label{eq:GHP_derivatives}
\begin{align}
	\Thorn\eta
	\equiv&
	\left(D-p\epsilon-q\oline{\epsilon}\right)\eta
	,\\
	\Thorn^{\prime}\eta
	\equiv&
	\left(\Delta-p\gamma-q\oline{\gamma}\right)\eta
	,\\
	\edth\eta
	\equiv&
	\left(\delta-p\beta-q\oline{\alpha}\right)\eta
	,\\
	\edth^{\prime}\eta
	\equiv&
	\left(\oline{\delta}-p\alpha-q\oline{\beta}\right)\eta
	.
\end{align}
\end{subequations}
	These operators have the definite spin and boost weight
\begin{center}
\begin{tabular}{ c c c c }
\hline
 operator & weight &  spin weight &  boost weight \\ 
\hline
$\Thorn$ 		& $\{ 1, 1\}$ & $ 0$ & $ 1$ \\ 
$\Thorn^{\prime}$ 	& $\{-1,-1\}$ & $ 0$ & $-1$ \\ 
$\edth$ 		& $\{ 1,-1\}$ & $ 1$ & $ 0$ \\ 
$\edth^{\prime}$ 	& $\{-1, 1\}$ & $-1$ & $ 0$ 
\end{tabular}
\end{center}

	Finally, we note that there is a physical
interpretation for spin weight: spin 0 captures ``scalar'' information,
spin 1 captures ``vector'' information, and spin 2 captures
``tensor'' information.

%%%%%%%%%%%%%%%%%%%%%%%%%%%%%%%%%%%%%%%%%%%%%%%%%%%%%%%%%%%%%%%%%%%%%%%%%%%%%%
\section{Interpretation of the NP scalars}
	We can interpret several of the NP scalars using the 
Raychaudhuri equations of null congruences of the $n^i$, $l^i$ null
vectors. These interpretations prove to be extremely useful when using
NP equations.

%%%%%%%%%%%%%%%%%%%%%%%%%%%%%%%%%%%%%%%%%%%%%%%%%%%%%%%%%%%%%%%%%%%%%%%%%%%%%%
\subsection{Derivatives of NP vectors}
	First we write out
\begin{align}
	\nabla_ie_{(a)j}
	=
	e^{(b)}_ie^k_{(b)}\nabla_je_{(a)j}=\gamma_{(c)(a)(b)}e^{(b)}_ie^{(c)}_j
	.
\end{align}
	From this and the NP relations, we have
\begin{subequations}
\label{eq:der_np_vecs}
\begin{align}
	\nabla_il_j
	= &
	-\left(l_j\oline{m}_i\left(\oline{\alpha} + \beta\right)\right) - l_jm_i\left(\alpha + \oline{\beta}\right) + l_il_j\left(\gamma + \oline{\gamma}\right) + l_jn_i\left(\epsilon + \oline{\epsilon}\right) - \oline{m}_jn_i\kappa 
	\nonumber \\ &
	- m_jn_i\oline{\kappa} + \oline{m}_jm_i\rho + \oline{m}_im_j\oline{\rho} + \oline{m}_i\oline{m}_j\sigma + m_im_j\oline{\sigma} - l_i\oline{m}_j\tau - l_im_j\oline{\tau}
	, \\
	\nabla_in_j
	= &
	m_in_j\alpha + \oline{m}_in_j\oline{\alpha} + \oline{m}_in_j\beta + m_in_j\oline{\beta} - l_in_j\gamma - l_in_j\oline{\gamma} - n_in_j\epsilon - n_in_j\oline{\epsilon} - m_im_j\lambda  
	\nonumber \\ &
	- \oline{m}_i\oline{m}_j\oline{\lambda} - \oline{m}_im_j\mu - \oline{m}_jm_i\oline{\mu} + l_im_j\nu + l_i\oline{m}_j\oline{\nu} + m_jn_i\pi + \oline{m}_jn_i\oline{\pi}
	, \\
	\nabla_im_j
	= &
	-\left(m_im_j\alpha\right) + \oline{m}_im_j\oline{\alpha} - \oline{m}_im_j\beta + m_im_j\oline{\beta} + l_im_j\gamma - l_im_j\oline{\gamma} + m_jn_i\epsilon - m_jn_i\oline{\epsilon} 
	\nonumber \\ &
	- n_in_j\kappa - l_j\oline{m}_i\oline{\lambda} - l_jm_i\oline{\mu} + l_il_j\oline{\nu} + l_jn_i\oline{\pi} + m_in_j\rho + \oline{m}_in_j\sigma - l_in_j\tau
	. 
\end{align}
\end{subequations}
%%%%%%%%%%%%%%%%%%%%%%%%%%%%%%%%%%%%%%%%%%%%%%%%%%%%%%%%%%%%%%%%%%%%%%%%%%%%%%
\subsection{Geodesics}
	From the equations for $l$ and $n$ we have
\begin{align}
	l^i\nabla_il_j
	= &
	\left(\epsilon+\oline{\epsilon}\right)l_j
-	\kappa \oline{m}_j
-	\oline{\kappa} m_j
	\nonumber \\
	n^i\nabla_in_j
	= &
-	\left(\gamma+\oline{\gamma}\right)n_j
+	\nu m_j
+	\oline{\nu} \oline{m}_j
	.
\end{align}
	We see for $l^i$/$n^i$ to be pregeodesic we need
$\kappa=0$/$\nu=0$. If we want $l^i$/$n^i$ to be geodesic we
additionally need $\mathfrak{R}\epsilon=0$/$\mathfrak{R}\gamma=0$.

%%%%%%%%%%%%%%%%%%%%%%%%%%%%%%%%%%%%%%%%%%%%%%%%%%%%%%%%%%%%%%%%%%%%%%%%%%%%%%
\section{Petrov classification}
	From the 
Eqs.~\eqref{eq:null_frame_transformations_class_one_Weyl_scalars},
\eqref{eq:null_frame_transformations_class_two_Weyl_scalars},
and \eqref{eq:null_frame_transformations_class_three_Weyl_scalars}, we see
that at any given tangent space it may be possible to set some of the
Weyl scalars to zero through a choice of null frame transformation. From
the perspective of the NP formalism, 
we can think of the  Petrov classification of spacetimes as enumerating 
how many Weyl scalars can be sent to zero through a null
frame transformation.
%%%%%%%%%%%%%%%%%%%%%%%%%%%%%%%%%%%%%%%%%%%%%%%%%%%%%%%%%%%%%%%%%%%%%%%%%%%
\chapter{Coordinates and null tetrads for Kerr}
%%%%%%%%%%%%%%%%%%%%%%%%%%%%%%%%%%%%%%%%%%%%%%%%%%%%%%%%%%%%%%%%%%%%%%%%%%%
These are collected equations on various tetrads and coordinates
for the Kerr spacetimes.
We begin with the Kinnersley tetrad, and by the end of the chapter
we have a tetrad that is regular both at the black hole horizon and
future null infinity. See also \cite{Ripley:2020xby}.
The black hole mass is $M$, and the black hole spin is $a$.
{\bf NOTE:} we use the non-standard symbol ``varpi'' $\varpi=3.14...$,
and reserve use of $\pi$ for the Newman-Penrose scalar.
%%%%%%%%%%%%%%%%%%%%%%%%%%%%%%%%%%%%%%%%%%%%%%%%%%%%%%%%%%%%%%%%%%%%%%%%%%%
\section{Kerr in Boyer-Lindquist coordinates}
%%%%%%%%%%%%%%%%%%%%%%%%%%%%%%%%%%%%%%%%%%%%%%%%%%%%%%%%%%%%%%%%%%%%%%%%%%%%%%
\subsection{Setup}
	The Kerr spacetime in Boyer-Lindquist coordinates is
\begin{align}
\label{eq:Kerr_BL_coords}
	ds^2
	=
	\left(1-\frac{2Mr}{\Sigma_{BL}}\right)dt^2
+	2\left(\frac{2Mar\mathrm{sin}^2\vartheta}{\Sigma_{BL}}\right)dtd\varphi
-	\frac{\Sigma_{BL}}{\Delta_{BL}}dr^2
	\nonumber \\
-	\Sigma_{BL}d\vartheta^2
-	\mathrm{sin}^2\vartheta\left(
		r^2+a^2+2Ma^2r\frac{\mathrm{sin}^2\vartheta}{\Sigma_{BL}}
	\right)
	d\varphi^2
	,
\end{align}
	where
\begin{subequations}
\begin{align}
	\Sigma_{BL}\equiv & r^2+a^2\mathrm{cos}^2\vartheta 
	, \\
	\Delta_{BL}\equiv & r^2-2Mr+a^2
	.
\end{align}
\end{subequations}
%%%%%%%%%%%%%%%%%%%%%%%%%%%%%%%%%%%%%%%%%%%%%%%%%%%%%%%%%%%%%%%%%%%%%%%%%%%%%%
\subsection{The tetrad}
	The Kinnersley tetrad in Boyer-Lindquist coordinates is
\begin{subequations}
\label{eq:Kinnersley_tetrad_BL}
\begin{align}
	l_{Kin}^{i}
	= &
	\left(\frac{r^2+a^2}{\Delta_{BL}},1,0,\frac{a}{\Delta_{BL}}\right)
	, \\
	n_{Kin}^{i}
	= &
	\frac{1}{2\Sigma_{BL}}\left(r^2+a^2,-\Delta_{BL},0,a\right)
	, \\
	m_{Kin}^{i}
	= i&
	\frac{1}{2^{1/2}\left(r+ia\mathrm{cos}\vartheta\right)}
	\left(
		ia\mathrm{sin}\vartheta,0,1,\frac{i}{\mathrm{sin}\vartheta}
	\right)
	.
\end{align}
\end{subequations}
	The event horizon is at $(\nabla r)^2=0$;
that is at the point $\Delta_{BL}=0$. This function has two zeros, which
define the outer and inner horizons
\begin{subequations}
\begin{align}
	r_{ou} 
	= & 
	M \pm \left(M^2-a^2\right)^{1/2}
	.
\end{align}
\end{subequations}
%%%%%%%%%%%%%%%%%%%%%%%%%%%%%%%%%%%%%%%%%%%%%%%%%%%%%%%%%%%%%%%%%%%%%%%%%%%
\section{Kerr in ingoing Eddington-Finkelstein coordinates}
%%%%%%%%%%%%%%%%%%%%%%%%%%%%%%%%%%%%%%%%%%%%%%%%%%%%%%%%%%%%%%%%%%%%%%%%%%%%%%
\subsection{Setup}
	We transform
\begin{subequations}
\begin{align}
	dv 
	\equiv & 
	dt + dr_* - dr 
	, \\
	d\phi 
	\equiv & 
	d\varphi + \frac{a}{r^2+a^2}dr_*
	.
\end{align}
\end{subequations}
	where
\begin{align}
	\frac{dr_*}{dr} 
	\equiv
	\frac{r^2+a^2}{\Delta_{BL}}
	.
\end{align}
	The Kerr metric in ingoing Eddington-Finkelstein coordinates is
\begin{align}
	ds^2
	= &
	\left(1-\frac{2Mr}{\Sigma_{BL}}\right)dv^2
-	\frac{4Mr}{\Sigma_{BL}}\left(
		dr
	-	a\mathrm{sin}^2\vartheta d\phi
	\right)
	dv
	\nonumber \\ &
-	\left(1+\frac{2Mr}{\Sigma_{BL}}\right)
	\left(
		dr^2
	-	2a\mathrm{sin}^2\vartheta drd\phi
	\right)
	\nonumber \\ &
-	\Sigma d\vartheta^2
-	\left(
		a^2+r^2+2Mr\frac{a^2}{\Sigma_{BL}}\mathrm{sin}^2\vartheta
	\right)
	d\phi^2
	.
\end{align}
%%%%%%%%%%%%%%%%%%%%%%%%%%%%%%%%%%%%%%%%%%%%%%%%%%%%%%%%%%%%%%%%%%%%%%%%%%%%%%
\subsection{The tetrad}
	The Kinnersely tetrad transforms as
\begin{align}
	v^i\to \frac{\partial x^i}{\partial y^j}v^j
	,
\end{align}
	so that
\begin{subequations}
\label{eq:Kinnersley_tetrad_IEF}
\begin{align}
	l_{Kin}^{i}
	= &
	\left(
		1+\frac{4Mr}{\Delta_{BL}},
		1,
		0,
		\frac{2a}{\Delta_{BL}}
	\right)
	, \\
	n_{Kin}^{i}
	= &
	\frac{1}{2\Sigma_{BL}}\left(
		\Delta_{BL},
		-\Delta_{BL},
		0,
		0
	\right)
	, \\
	m_{Kin}^{i}
	= &
	\frac{1}{2^{1/2}\left(r+ia\mathrm{cos}\vartheta\right)}
	\left(
		ia\mathrm{sin}\vartheta,0,1,\frac{i}{\mathrm{sin}\vartheta}
	\right)
	.
\end{align}
\end{subequations}
	This tetrad is singular at the black hole horizons.
This coordinate singularity
can be removed by the following tetrad transformation \cite{Teukolsky:1973ha}
\begin{subequations}
\begin{align}
	l^i\to \Delta_{BL}l^i, \\
	n^i\to \frac{1}{\Delta_{BL}}n^i
	,
\end{align}
\end{subequations} 
	but this spoils the property that $\epsilon=0$ (which holds in
the Kinnersely tetrad). We rotate the tetrad to set $\gamma=0$, which
is more useful for metric reconstruction in outgoing radiation gauge
\begin{subequations}
\begin{align}
	l^i
	\to &
	\frac{\Delta_{BL}}{2\Sigma_{BL}}l^i
	, \\
	n^i
	\to &
	\frac{2\Sigma_{BL}}{\Delta_{BL}}n^i
	, \\
	m^i
	\to &
	\mathrm{exp}\left[-2i\mathrm{arctan}\left[\frac{r}{a\mathrm{sin}\vartheta}\right]\right]m^i
	,
\end{align}
\end{subequations}
	we obtain
\begin{subequations}
\label{eq:new_tetrad_EF}
\begin{align}
	l^i
	= &
	\left(
		\frac{r^2+2Mr+a^2}{2\Sigma_{BL}},
		\frac{\Delta_{BL}}{2\Sigma_{BL}},
		0,
		\frac{a}{2\Sigma_{BL}}
	\right)
	, \\
	n^i
	= &
	\left(
		1,-1,0,0
	\right)
	, \\
	m^i
	= &
	\frac{1}{2^{1/2}\left(r-ia\mathrm{cos}\vartheta\right)}
	\left(
		-ia\mathrm{sin}\vartheta,0,-1,-\frac{i}{\mathrm{sin}\vartheta}
	\right)
	.
\end{align}
\end{subequations}
%%%%%%%%%%%%%%%%%%%%%%%%%%%%%%%%%%%%%%%%%%%%%%%%%%%%%%%%%%%%%%%%%%%%%%%%%%%
\section{Kerr in ingoing Eddington-Finkelstein coordinates with
hyperboloidal compactification}
%%%%%%%%%%%%%%%%%%%%%%%%%%%%%%%%%%%%%%%%%%%%%%%%%%%%%%%%%%%%%%%%%%%%%%%%%%%
\subsection{Setup}
	For further discussion of hyperboloidal compactification see, e.g.
\cite{Zenginoglu_2008}.
The ingoing/outgoing radial null characteristic speeds for
Kerr in ingoing Eddington-Finkelstein coordinates are
\begin{subequations}
\begin{align}
	c_+
	= &
	1
-	\frac{4Mr}{2Mr+\Sigma_{BL}}
	, \\
	c_-
	= &
	-1
	.
\end{align}
\end{subequations}
	We do not need to consider angular characteristic speeds as those
die off more quickly as we go to future null infinity/spatial infinity.
We choose a radial compactification and time rescaling
$R(r)$ and $T(v,r)$, respectively.
The ingoing/outgoing radial null characteristic speeds are now
\begin{align}
	\tilde{c}_{\pm}
	=
	\frac{dR/dr}{\frac{1}{c_{\pm}}\partial_vT+\partial_rT}
	.
\end{align}
	We want to choose a time function that sets
$\tilde{c}_-|_{r=\infty}=0$ while keeping $0<\tilde{c}_+|_{r=\infty}<\infty$.
We choose the time coordinate to be of the form
\begin{align}
	T(v,r) 
	= 
	v + h(r) 
	, 
\end{align}
	and a compactification
\footnote{We choose this compactification
as it is straightforward to manipulate.}
\begin{align}
	R(r) \equiv \frac{L^2}{r}
	,
\end{align}
	where $L$ is a constant.
Series expanding about $r=\infty$, we have
\begin{subequations}
\begin{align}
	\tilde{c}_+
	= &
	\left(
		1+\frac{4M}{r}+\frac{8M^2}{r^2}
	+	\mathcal{O}\left(\frac{1}{r^3}\right)
	+	\frac{dh}{dr}
	\right)^{-1}
	\left(-\frac{L^2}{r^2}\right)
	, \\
	\tilde{c}_-
	= &
	\left(-1+\frac{dh}{dr}\right)^{-1}
	\left(-\frac{L^2}{r^2}\right)	
	.
\end{align}
\end{subequations}
	We see that the choice
\begin{align}
	\frac{dh}{dr} = -1 - \frac{4M}{r}
	,
\end{align}
	sets
$\tilde{c}_-|_{R=0}=0$ while keeping
$0>\tilde{c}_+|_{R=0}>-\infty$ (our choice of compactification
flips the signs of the ingoing and outgoing characteristics, and
$r=\infty$ is mapped to $R=0$). We can say that $R=0$ is located
``on'' future null infinity of Kerr.

%%%%%%%%%%%%%%%%%%%%%%%%%%%%%%%%%%%%%%%%%%%%%%%%%%%%%%%%%%%%%%%%%%%%%%%%%%%
\subsection{The tetrad}
We apply the above transformations $T(v,r)$, $R(r)$,
to Eq.~\eqref{eq:new_tetrad_EF} to obtain
\begin{subequations}
\label{eq:tetrad_IEF_HC}
\begin{align}
	l^i 
	= &
	\frac{R^2}{L^4+a^2R^2\mathrm{cos}^2\vartheta}\left(
		2M\left(2M-\left(\frac{a}{L}\right)^2R\right),
	-	\frac{1}{2}\left(L^2-2MR+\left(\frac{a}{L}\right)^2R^2\right),
		0,
		a
	\right)
	, \\
	n^i 
	= &
	\left(
		2+\frac{4MR}{L^2},\frac{R^2}{L^2},0,0
	\right)
	, \\
	m^i
	= &
	\frac{R}{2^{1/2}\left(L^2-iaR\mathrm{cos}\vartheta\right)}
	\left(
		-ia\mathrm{sin}\vartheta,
		0,
		-1,
		-\frac{i}{\mathrm{sin}\vartheta}
	\right)
	.
\end{align}
\end{subequations}
%%%%%%%%%%%%%%%%%%%%%%%%%%%%%%%%%%%%%%%%%%%%%%%%%%%%%%%%%%%%%%%%%%%%%%%%%%%
\subsection{Weyl scalars and Ricci rotation coefficients}
	The nonzero Weyl scalar is
\begin{align}
	\Psi_2
	=
	-\frac{
		MR^3
	}{
		\left(L^2 - iaR\mathrm{cos}\left(\vartheta\right)\right)^3
	}
	.
\end{align}
	and the nonzero Ricci rotation coefficients are
\begin{subequations}
\label{eq:NP_IEF_HC}.
\begin{align}
	\rho
	= &
	-\frac{R \left(a^2 R^2+L^4-2 L^2 M R\right)}{2 \left(L^2-i a R \cos (\vartheta )\right)^2 \left(L^2+i a R \cos (\vartheta )\right)}
	, \\
	\mu
	= &
	\frac{R}{-L^2+i a R \cos (\vartheta )}
	, \\
	\tau
	= &
	\frac{i a R^2 \sin (\vartheta )}{\sqrt{2} \left(L^2-i a R \cos (\vartheta )\right)^2}
	, \\
	\pi
	= &
	-\frac{i a R^2 \sin (\vartheta )}{\sqrt{2} \left(a^2 R^2 \cos ^2(\vartheta )+L^4\right)}
	, \\
	\epsilon
	= &
	\frac{R^2 \left(a^2 (-R)-i a \cos (\vartheta ) \left(L^2-M R\right)+L^2 M\right)}{2 \left(L^2-i a R \cos (\vartheta )\right)^2 \left(L^2+i a R \cos (\vartheta )\right)}
	, \\
	\alpha
	= &
	\frac{R \cot (\vartheta )}{\sqrt{2} \left(2 L^2+2 i a R \cos (\vartheta )\right)}
	, \\
	\beta
	= &
	\frac{R \left(-L^2 \cot (\vartheta )+i a R \sin (\vartheta ) \left(\csc ^2(\vartheta )+1\right)\right)}{2 \sqrt{2} \left(L^2-i a R \cos (\vartheta )\right)^2}
	.
\end{align}
\end{subequations}
	These are all regular on the black hole horizon and future null
infinity ($R=0$). Some of the rotation coefficients are singular
at the poles $\vartheta=0,\varpi$.
%%%%%%%%%%%%%%%%%%%%%%%%%%%%%%%%%%%%%%%%%%%%%%%%%%%%%%%%%%%%%%%%%%%%%%%%%%%%%%
%%%%%%%%%%%%%%%%%%%%%%%%%%%%%%%%%%%%%%%%%%%%%%%%%%%%%%%%%%%%%%%%%%%%%%%%%%%%%%
\bibliography{jripley_notes_bib}
\bibliographystyle{alpha}

\end{document}

