\documentclass{my_cv}
\usepackage[margin=0.5in]{geometry}
\usepackage{amssymb}
\usepackage{etaremune}
\begin{document}
%=============================================================================
\name{Justin L. Ripley}
\bigskip
\contact{
   DAMTP, University of Cambridge
}{	
   Wilberforce Road, Cambridge CB3 0WA, UK
}{
   lloydripley@gmail.com
}{
   \url{https://jlripley314.github.io/}
}{
   (619)-851-1226
}
%=============================================================================
\section{Employment}
%-----------------------------------------------------------------------------
\subsection{Research Associate}
\begin{itemize}
\dateditem{DAMTP, University of Cambridge}{October 2020--present}
\end{itemize}
%-----------------------------------------------------------------------------
\subsection{Graduate student Research Assistant and Assistant Instructor}
\begin{itemize}
\dateditem{Department of Physics, Princeton University}{September 2014-- July 2020}
\end{itemize}
%=============================================================================
\section{Education}
%-----------------------------------------------------------------------------
\datedsubsection{Princeton University}{2014--2020}
\begin{itemize}
\item Advisor: Frans Pretorius
\item PhD  Physics, 2020 
\item M.A. Physics, 2016
\end{itemize}
%-----------------------------------------------------------------------------
\datedsubsection{Columbia University}{2010--2014}
\begin{itemize}
\item B.A. Physics, 2014
\end{itemize}
%=============================================================================
\section{Research Interests}
	General relativity (GR), and modeling of
astrophysical and cosmological phenomena which can be used
to test and increase our understanding of the dynamics of GR and the 
Standard Model of particle physics.
This research program includes work in
the mathematics of GR and of modified gravity theories, to better understand
the self-consistency of modeling used to compare theory to observation and
experiment.
I am also interested in numerical relativity, numerical analysis, and
scientific visualization.
%=============================================================================
\section{Awards/Grants}
%-----------------------------------------------------------------------------
\subsection{International Society on General Relativity and Gravitation}
\begin{itemize}
\dateditem{Hartle award for best talk by a student (GR 22/Amaldi 13 conference)}{December 2019}
\end{itemize}
%-----------------------------------------------------------------------------
\subsection{National Science Foundation}
\begin{itemize}
\dateditem{GRFP honorable mention}{March 2015}
\end{itemize}
%-----------------------------------------------------------------------------
\subsection{Columbia University}
\begin{itemize}
\dateditem{\emph{summa cum laude}, Phi Beta Kappa, Departmental honors in physics}{May 2014}
\dateditem{Erwin H. Leiwant Scholarship}{September 2013--May 2014}
\dateditem{John Jay Scholar}{September 2010--May 2014}
\end{itemize}
%=============================================================================
\section{Refereed Publications}
%-----------------------------------------------------------------------------
%Preprints may be found on 
%\href{https://arxiv.org/search/?query=Ripley%2C+Justin+L&searchtype=author&abstracts=show&order=-announced_date_first&size=50}{arXiv},
%\href{https://inspirehep.net/authors/1477964}{InSpire Hep},
%\href{https://ui.adsabs.harvard.edu/search/p_=0&q=author%3A%22Ripley%2C%20Justin%20L.%22&sort=date%20desc%2C%20bibcode%20desc}{NASA/ADS},
%or
%\href{https://scholar.google.com/citations?user=a7k5tZ8AAAAJ&hl=en}{Google Scholar}.
%-----------------------------------------------------------------------------
\begin{etaremune}
\item {\bf Justin L. Ripley}, Frans Pretorius 
	\emph{Dynamics of a $\mathbb{Z}_2$ symmetric EdGB gravity in
	spherical symmetry}.
	Class. Quant. Grav. 37 (15), 155003.
	arXiv:2005.05417
%-----------------------------------------------------------------------------
\item {\bf Justin L. Ripley}, Frans Pretorius 
	\emph{Scalarized black hole dynamics in
	Einstein-dilaton-Gauss-Bonnet gravity}.
	Phys. Rev. D 101 (4), 044015.
	arXiv:1911.11027
%-----------------------------------------------------------------------------
\item {\bf Justin L. Ripley}, 
	\emph{Excision and avoiding the use of boundary conditions
	in numerical relativity}.
	Class. Quantum Grav. 36 (23) 237001.  
	arXiv:1908.04234
%-----------------------------------------------------------------------------
\item {\bf Justin L. Ripley}, Frans Pretorius, 
	\emph{Gravitational collapse in Einstein
	dilaton Gauss-Bonnet gravity}
	Class. Quantum Grav. 36 (13) 134001. arXiv:1903.07543
	(Invited to Focus Issue on Numerical
	Relativity Beyond General Relativity)
%-----------------------------------------------------------------------------
\item {\bf Justin L. Ripley}, Frans Pretorius, 
	\emph{Hyperbolicity in Spherical Collapse of a Horndeski Theory.}
	Phys. Rev. D 99 (8), 084014. arXiv:1902.01468
%-----------------------------------------------------------------------------
\item {\bf Justin L. Ripley}, Kent Yagi, 
	\emph{Black hole perturbation under a 2+2 decomposition
	in the action.}
	Phys. Rev. D 97 (2), 024009. arXiv:1705.03068
%-----------------------------------------------------------------------------
\item Anna Ijjas, {\bf Justin L. Ripley}, Paul J. Steinhardt,
	\emph{NEC violation in mimetic cosmology revisited.}
	Phys.Lett. B760 132-138. arXiv:1604.08586
%-----------------------------------------------------------------------------
\item {\bf Justin L. Ripley}, Brian D. Metzger,	
	Almudena Arcones, and Gabriel Martnez-Pinedo,
	\emph{X-ray Decay Lines from Heavy Nuclei in
	Supernova Remnants as a Probe of the r-Process Origin
	and the Birth Periods of Magnetars.}
	Mon. Not. Roy. Astron. Soc. 438 (4), 3243-3254.
	arXiv:1310.2950
\end{etaremune}
%=============================================================================
\section{Invited Talks}
%-----------------------------------------------------------------------------
\begin{itemize}
%-----------------------------------------------------------------------------
\datedtalk{Perimeter Institute, Waterloo, ON (virtual talk)}{
	Exploring the nonlinear dynamics
         of Einstein dilaton Gauss-Bonnet gravity}{
	April 2020}
%-----------------------------------------------------------------------------
\datedtalk{University of Illinois, Urbana-Champaign, IL}{
	Testing General Relativity and the nonlinear dynamics of modified
	gravity theories}{
	January 2020}
%-----------------------------------------------------------------------------
\datedtalk{Black Hole Initiative, Harvard University, Cambridge, MA}{
	Nonlinear dynamics of Horndeski theories in spherical collapse}{
	December 2019}
\end{itemize}
%=============================================================================
\section{Seminars/Contributed Talks}
%-----------------------------------------------------------------------------
\begin{itemize}
%-----------------------------------------------------------------------------
\datedtalk{APS April Meeting (virtual talk)}{
	Second order perturbation of a Kerr black hole}{
	April 2020}
%-----------------------------------------------------------------------------
\datedtalk{``Gravity Group'', Princeton University, Princeton, NJ}{
	Modeling the `ringdown' of a Kerr black hole}{
	March 2020}
%-----------------------------------------------------------------------------
\datedtalk{Massachusetts Institute of Technology, Cambridge, MA}{
	Second order vacuum perturbation of a Kerr black hole}{
	December 2019}
%-----------------------------------------------------------------------------
\datedtalk{GR 22/Amaldi 13, Valencia, Spain}{
	Nonlinear dynamics of Horndeski theories in spherical collapse}{
	July 2019}
%-----------------------------------------------------------------------------
\datedtalk{APS April Meeting, Denver, CO}{
	Hyperbolicity in gravitational collapse in a modified gravity theory}{
	April 2019}
%-----------------------------------------------------------------------------
\datedtalk{Numerical Relativity beyond General Relativity, Benasque, Spain}{
	Gravitational collapse in a modified gravity theory}{
	June 2018}
\end{itemize}
%=============================================================================
\section{Computational Experience}
%-----------------------------------------------------------------------------
\begin{itemize}
\item Languages: C/C++, Fortran, Mathematica, Python 
\item \href{https://github.com/JLRipley314}{Link} to
      my Github account, which contains open-source code for
      some of the projects I have worked on.
\end{itemize}
%=============================================================================
\section{Professional Activities}
%-----------------------------------------------------------------------------
\subsection{Committees}
\begin{itemize}
\dateditem{Member of Climate and Inclusion Committee, \\
      Department of Physics, Princeton University}{
	September 2019--May 2020}
\end{itemize}
%-----------------------------------------------------------------------------
\subsection{Professional Organizations}
\begin{itemize}
\dateditem{Member of American Physical Society}{
	2018--present}
\end{itemize}
%-----------------------------------------------------------------------------
\subsection{Seminar Organizer}
\begin{itemize}
\dateditem{Friday GR seminar, \\
      DAMTP, University of Cambridge}{
	October 2020-present}
\end{itemize}
%-----------------------------------------------------------------------------
\subsection{Journal Referee}
\begin{itemize}
\item Physical Review D, Physical Review Letters
\end{itemize}
%=============================================================================
\section{Teaching and Mentorship}
%-----------------------------------------------------------------------------
\subsection{Assistant Instructor, Princeton University}
%-----------------------------------------------------------------------------
\begin{itemize}
\dateditem{EGR/PHY 191, An integrated introduction to engineering, math, physics}{Fall 2019}
\dateditem{PHY 103/105, General Physics I Lab}{Fall 2018}
\dateditem{PHY 304, Advanced Electromagnetism}{Spring 2018}
\dateditem{AST 203, The Universe}{Spring 2017,2018}
\dateditem{PHY 523, General Relativity}{Fall 2017}
\dateditem{AST 204, Topics in Modern Astronomy}{Spring 2016}
\dateditem{PHY 301, Thermal Physics}{Fall 2015, Spring 2016}
\end{itemize}
%-----------------------------------------------------------------------------
\subsection{Teaching Assistant, Columbia University}
\begin{itemize}
\dateditem{Math V2000, Introduction to higher mathematics}{Spring 2014}
\end{itemize}
%=============================================================================
\section{Outreach}
%-----------------------------------------------------------------------------
\subsection{Open Labs}
	Open Labs is a graduate student group at Princeton University
that organizes ``science cafes''
where local high and middle school students hear talks given by graduate
students about their research.
\begin{itemize}
\dateditem{Treasurer and active member}{May 2018--February 2019}
\end{itemize}
%-----------------------------------------------------------------------------
\subsection{Princeton Citizen Scientists}
	The Princeton Citizen Scientists is a graduate student led group
at Princeton University that is dedicated to science policy and
outreach at the local, state, and federal level. 
\begin{itemize}
\dateditem{President}{June 2018--July 2019}
\dateditem{Co-organizer for science advocacy trip to Washington, D.C;\\
	see this \href{http://www.dailyprincetonian.com/article/2019/01/justin-ripley-q-and-a}{article}
	in the Daily Princetonian}{
	December 2018}
\dateditem{Co-organizer for science ``teach-in'' event at Princeton Public Library; \\
	see this
	\href{http://www.dailyprincetonian.com/article/2017/10/princeton-citizen-scientists-host-teach-in-at-princeton-public-library}{article}
	in the Daily Princetonian}{
	October 2017}
\end{itemize}
%-----------------------------------------------------------------------------
\subsection{Interviews on ``These Vibes are Too Cosmic''}
	These Vibes are Too Cosmic is a radio program run through
	Princeton University.
\begin{itemize}
\dateditemlink{Interview on exotic compact objects}{
	https://tvr2c.com/2019/02/14/justinripley2ecos/}{
	January 2019}
\dateditemlink{Interview on antigravity}{
	https://tvr2c.com/2016/03/16/justincosmology/}{
	March 2016}
\end{itemize}
%=============================================================================
\end{document}
