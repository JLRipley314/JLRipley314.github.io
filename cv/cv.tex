\documentclass{my_cv}
\usepackage[margin=0.5in]{geometry}
\usepackage{amssymb}
\usepackage{etaremune}
\usepackage{indentfirst}
\begin{document}
%=============================================================================
\name{Justin L. Ripley}
\bigskip
\contact{
   DAMTP, University of Cambridge
}{	
   Wilberforce Road, Cambridge CB3 0WA, UK
}{
   lloydripley@gmail.com
}{
   \url{https://jlripley314.github.io/}
}{
   Citizenship: U.S.A.
}
%=============================================================================
\section{Academic Employment}
%-----------------------------------------------------------------------------
\noindent
\datedline{{\bf Research Associate}, 
   Department of Physics, University of Illinois, Urbana-Champaign
   }{
      August 2022-present
   }
\\ \\
%-----------------------------------------------------------------------------
\noindent
\datedline{{\bf Research Associate}, DAMTP, University of Cambridge}{October 2020-June 2022}
\\ \\
%-----------------------------------------------------------------------------
\datedline{{\bf Research and Teaching Assistant}, Princeton University}{September 2014-July 2020}
%=============================================================================
\section{Education}
%-----------------------------------------------------------------------------
\noindent
\datedline{{\bf PhD, Physics}, Princeton University}{September 2014-July 2020}
\indentline
   Advisor: Frans Pretorius
%-----------------------------------------------------------------------------
\\ \\
\datedline{{\bf B.A., Physics}, Columbia University}{September 2010-May 2014}
\indentline
   Minor in Mathematics
\indentline
   Departmental honors in Physics, \emph{summa cum laude}, Phi Beta Kappa
%=============================================================================
%\section{Research Interests}
%	General relativity (GR), and modeling of
%astrophysical and cosmological phenomena which can be used
%to test and increase our understanding of the dynamics of GR and the 
%Standard Model of particle physics.
%This research program includes work in
%the mathematics of GR and of modified gravity theories, to better understand
%the self-consistency of modeling used to compare theory to observation and
%experiment.
%I am also interested in numerical relativity, numerical analysis, and
%scientific visualization.
%=============================================================================
\section{Awards/Grants}
%-----------------------------------------------------------------------------
\noindent
\datedline{{\bf Hartle award}, ISGRG (GR 22/Amaldi 13 conference)}{December 2019}
%-----------------------------------------------------------------------------
\\ \\
\datedline{{\bf Erwin H. Leiwant Scholarship}, Columbia University}{September 2013-May 2014}
%-----------------------------------------------------------------------------
\\ \\
\datedline{{\bf John Jay Scholar}, Columbia University}{September 2010-May 2014}
%=============================================================================
\section{Computational Experience}
%-----------------------------------------------------------------------------
   I have programming experience with C/C++, Fortran (77/90), Julia, Python,
   and Mathematica.
   My Github account: \href{https://github.com/JLRipley314}{JLRipley314},
   lists some of the individual computational projects I have worked on.
   I have also done some work for the
   \href{https://www.grchombo.org/}{GRChombo collaboration}, 
   which works on an open-source numerical relativity code written in C++.
%=============================================================================
\section{Teaching and Mentorship}
%-----------------------------------------------------------------------------
\noindent
{\bf Assistant Instructor, Princeton University}
\\ \indent
%-----------------------------------------------------------------------------
\datedline{EGR/PHY 191, An integrated introduction to engineering, math, physics}{Fall 2019}
%-----------------------------------------------------------------------------
\\ \indent
\datedline{PHY 103/105, General Physics I Lab}{Fall 2018}
%-----------------------------------------------------------------------------
\\ \indent
\datedline{PHY 304, Advanced Electromagnetism}{Spring 2018}
%-----------------------------------------------------------------------------
\\ \indent
\datedline{AST 203, The Universe}{Spring 2017,2018}
%-----------------------------------------------------------------------------
\\ \indent
\datedline{PHY 523, General Relativity (graduate course)}{Fall 2017}
%-----------------------------------------------------------------------------
\\ \indent
\datedline{AST 204, Topics in Modern Astronomy}{Spring 2016}
%-----------------------------------------------------------------------------
\\ \indent
\datedline{PHY 301, Thermal Physics}{Fall 2015, Spring 2016}
%-----------------------------------------------------------------------------
\\ \\ \noindent
{\bf Teaching Assistant, Columbia University}
\\ \indent
\datedline{Math V2000, Introduction to higher mathematics}{Spring 2014}
%-----------------------------------------------------------------------------
\\ \\ \noindent
{\bf Supervisor for undergraduate summer student projects, 
University of Cambridge}
%\\ \indent
%Both projects received \emph{Faculty Summer Research in Maths} funding
%from the University of Cambridge.
 \\
\indent
\datedline{Shikhar Kumar, 
   \emph{Computing null geodesics in slightly perturbed black hole spacetimes}
   }{Summer 2021}
\\ \indent
\datedline{Adam Wills (co-supervised),
   \emph{Computing the quasinormal modes of wormholes}
   }{Summer 2021}
\pagebreak
%=============================================================================
\section{Professional Activities}
%-----------------------------------------------------------------------------
\noindent {\bf University of Cambridge, DAMTP}
\\ 
\datedline{Friday general relativity seminar co-organizer}{
   October 2020-June 2022
}
\\ 
\datedline{General relativity journal club co-organizer}{
   October 2020-June 2022
}
\\ 
%-----------------------------------------------------------------------------
\\ \noindent {\bf Princeton University Department of Physics}
\\
\datedline{Member on the
Climate and Inclusion Committee}{September 2019-May 2020}
\\
%-----------------------------------------------------------------------------
\\ \noindent {\bf Referee}
\\
\datedline{
   Physical Review D, 
   Physical Review Letters, 
   Classical and Quantum Gravity}{}
%\\
%-----------------------------------------------------------------------------
%\datedline{Member of American Physical Society}{
%	2018-present}
%=============================================================================
\section{Outreach}
%-----------------------------------------------------------------------------
\noindent {\bf Princeton citizen scientists}
\\ \indent
	The \href{https://citizenscientists.princeton.edu/}{Princeton Citizen Scientists} 
is a graduate student led group
at Princeton University that is dedicated to science policy and
outreach at the local, state, and federal level.
\\ \indent
\datedline{President}{June 2018--July 2019}
\\ \indent
\datedline{Co-organizer for science advocacy trip to Washington, D.C.
\href{http://www.dailyprincetonian.com/article/2019/01/justin-ripley-q-and-a}{(article)}}{December 2018}
\\ \indent
\datedline{Co-organizer for science and intersectionality workshop 
\href{https://citizenscientists.princeton.edu/science-and-intersectionality/}{(link to schedule)}}{February 2018}
\\ \indent
\datedline{Co-organizer for science ``teach-in'' event at Princeton public library
\href{http://www.dailyprincetonian.com/article/2017/10/princeton-citizen-scientists-host-teach-in-at-princeton-public-library}{(article)}}{October 2017}
\\ \\
%-----------------------------------------------------------------------------
\noindent {\bf Open labs}
\\ \indent 
   Open labs is a graduate student group at Princeton University
that organizes ``science cafes''
where local high and middle school students hear talks given by graduate
students about their research.
\\ \indent
\datedline{Treasurer and presenter}{May 2018--February 2019}
\\ \\ 
%-----------------------------------------------------------------------------
\noindent {\bf Department of physics, Princeton University}
\\ \indent
I participated in several science outreach events organized through
the department of physics at Princeton University throughout my time
as a graduate student.
events where I helped plan/organize some of programming are listed below.
\\ \indent
\datedline{Trenton science summer camp (helped plan and run several lessons over 2 weeks)}{July 2018}
\\ \\
%-----------------------------------------------------------------------------
\noindent {\bf Interviews on ``these vibes are too cosmic''}
\\ \indent
   These vibes are too cosmic is a radio program run through
   Princeton University.
\\ \indent
\datedline{\href{https://tvr2c.com/2019/02/14/justinripley2ecos/}{Interview about exotic compact objects}}{January 2019}
\\ \indent
\datedline{\href{https://tvr2c.com/2016/03/16/justincosmology/}{Interview about antigravity}}{March 2016}
%=============================================================================
\section{Refereed Publications}
%-----------------------------------------------------------------------------
Link to all papers, including
preprints: \href{https://inspirehep.net/authors/1477964}{InSpire Hep}
%-----------------------------------------------------------------------------
\begin{etaremune}
%-----------------------------------------------------------------------------
\item {\bf Justin L. Ripley} 
   \emph{Computing the quasinormal modes and eigenfunctions for the 
   Teukolsky equation using 
   horizon penetrating, hyperboloidally compactified coordinates}.
   arXiv:2202.03837
%-----------------------------------------------------------------------------
\item William E. East, {\bf Justin L. Ripley} 
   \emph{Dynamics of Spontaneous Black Hole Scalarization and Mergers
   in Einstein-Scalar-Gauss-Bonnet Gravity}.
   Phys. Rev. Lett. 127, 101102 (2021).
   arXiv:2105.08571
%-----------------------------------------------------------------------------
\item {\bf Justin L. Ripley}, 
   \emph{A symmetric hyperbolic formulation of the vacuum
      Einstein equations in affine-null coordinates}.
   Journal of Mathematical Physics 62, 062501 (2021).
   arXiv:2104.09972
%-----------------------------------------------------------------------------
\item {\bf Justin L. Ripley}, 
   Nicholas Loutrel, Elena Giorgi, and Frans Pretorius 
   \emph{Numerical computation of second-order vacuum perturbations of
      Kerr black holes}.
   Phys. Rev. D 103 (10), 104018 (2021). 
   arXiv:2010.00162
%-----------------------------------------------------------------------------
\item Nicholas Loutrel, {\bf Justin L. Ripley}, 
   Elena Giorgi, and Frans Pretorius 
   \emph{Second Order Perturbations of Kerr Black Holes: Reconstruction of
      the Metric}.
   Phys. Rev. D 103, 104017 (2021).
   arXiv:2008.11770
%-----------------------------------------------------------------------------
\item William E. East, {\bf Justin L. Ripley} 
   \emph{Evolution of Einstein-scalar-Gauss-Bonnet gravity
      using a modified harmonic formulation}.
   Phys.Rev.D 103 4, 044040 (2021).
   arXiv:2011.03547
%-----------------------------------------------------------------------------
\item {\bf Justin L. Ripley}, Frans Pretorius 
   \emph{Dynamics of a $\mathbb{Z}_2$ symmetric EdGB gravity in
      spherical symmetry}.
   Class. Quant. Grav. 37 (15), 155003 (2020).
   arXiv:2005.05417
%-----------------------------------------------------------------------------
\item {\bf Justin L. Ripley}, Frans Pretorius 
   \emph{Scalarized black hole dynamics in
      Einstein-dilaton-Gauss-Bonnet gravity}.
   Phys. Rev. D 101 (4), 044015 (2019).
   arXiv:1911.11027
%-----------------------------------------------------------------------------
\item {\bf Justin L. Ripley}, 
   \emph{Excision and avoiding the use of boundary conditions
      in numerical relativity}.
   Class. Quantum Grav. 36 (23) 237001 (2019).  
   arXiv:1908.04234
%-----------------------------------------------------------------------------
\item {\bf Justin L. Ripley}, Frans Pretorius, 
   \emph{Gravitational collapse in Einstein
      dilaton Gauss-Bonnet gravity}
   Class. Quantum Grav. 36 (13) 134001 (2019).
   arXiv:1903.07543
%-----------------------------------------------------------------------------
\item {\bf Justin L. Ripley}, Frans Pretorius, 
   \emph{Hyperbolicity in Spherical Collapse of a Horndeski Theory.}
   Phys. Rev. D 99 (8), 084014 (2019).
   arXiv:1902.01468
%-----------------------------------------------------------------------------
\item {\bf Justin L. Ripley}, Kent Yagi, 
   \emph{Black hole perturbation under a 2+2 decomposition
      in the action.}
   Phys. Rev. D 97 (2), 024009 (2017).
   arXiv:1705.03068
%-----------------------------------------------------------------------------
\item Anna Ijjas, {\bf Justin L. Ripley}, Paul J. Steinhardt,
   \emph{NEC violation in mimetic cosmology revisited.}
   Phys.Lett. B760 132-138 (2016).
   arXiv:1604.08586
%-----------------------------------------------------------------------------
\item {\bf Justin L. Ripley}, Brian D. Metzger,	
      Almudena Arcones, and Gabriel Martinez-Pinedo,
   \emph{X-ray Decay Lines from Heavy Nuclei in
      Supernova Remnants as a Probe of the r-Process Origin
      and the Birth Periods of Magnetars.}
   Mon. Not. Roy. Astron. Soc. 438 (4), 3243-3254 (2013).
   arXiv:1310.2950
\end{etaremune}

%=============================================================================
{\bf GRChombo collaboration papers:} Since 2020 I have been a member of the 
\href{https://www.grchombo.org/}{GRChombo collaboration}.
%-----------------------------------------------------------------------------
\begin{etaremune}
\item Radia et al., 
   \emph{Lessons for adaptive mesh refinement in numerical relativity.}
    Class. Quant. Grav. 39 (13) 135006 (2022).
    arXiv:2112.10567

\item Andrade et al.,
   \emph{GRChombo: An adaptable numerical relativity code for fundamental physics.}
   J. Open Source Softw. 6 (2021) 3703. 
   arXiv:2201.03458

\end{etaremune}
%=============================================================================
\section{Conferences and Seminars}
{\bf Invited conference talks/seminars}
\begin{etaremune}
%-----------------------------------------------------------------------------
\datedtalk{Black Hole Initiative, Harvard University, Cambridge, MA (online)}{
   Numerical Relativity and testing General Relativity with
   gravitational waves: Parts I\&II 
   }{
   March 2022}
%-----------------------------------------------------------------------------
\datedtalk{University of T\"{u}bingen, T\"{u}bingen, DE (online)}{
   Evolution of binary black hole systems in scalar Gauss-Bonnet gravity 
   }{
   February 2022}
%-----------------------------------------------------------------------------
\datedtalk{Albert Einstein Institute, Potsdam, DE (online)}{
   Evolution of binary black hole systems in scalar Gauss-Bonnet gravity 
   }{
   November 2021}
%-----------------------------------------------------------------------------
\datedtalk{Sapienza University of Rome, Rome, IT (online)}{
   Computing the second order gravitational perturbation of Kerr black holes
   }{
   May 2021}
%-----------------------------------------------------------------------------
\datedtalk{University of Oxford, Oxford, UK (online)}{
   The classical evolution of binary black hole
   systems in scalar-tensor theories}{
   February 2021}
%-----------------------------------------------------------------------------
\datedtalk{University of Virginia, Charlottesville, VA (online)}{
   The classical evolution of binary black hole
   systems in scalar-tensor theories}{
   February 2021}
%-----------------------------------------------------------------------------
\datedtalk{Kyoto University, Kyoto, JP (online)}{
   The classical evolution of binary black hole
   systems in scalar-tensor theories}{
   February 2021}
%-----------------------------------------------------------------------------
\datedtalk{University of Southampton, Southampton, UK (online)}{
   The classical evolution of binary black hole
   systems in scalar-tensor theories}{
   January 2021}
%-----------------------------------------------------------------------------
\datedtalk{University of Cambridge, Cambridge, UK (online)}{
   Computing the second order gravitational perturbation
   of Kerr black holes}{
   November 2020}
%-----------------------------------------------------------------------------
\datedtalk{Johns Hopkins University, Baltimore, MD (online)}{
   Numerical computation of second order vacuum perturbations of Kerr black holes}{
   November 2020}
%-----------------------------------------------------------------------------
\datedtalk{Princeton University, Princeton, NJ (online)}{
   Classical modifications to
   Einstein's General Relativity around black holes}{
   October 2020}
%-----------------------------------------------------------------------------
\datedtalk{Perimeter Institute, Waterloo, ON (online)}{
   Exploring the nonlinear dynamics
   of Einstein dilaton Gauss-Bonnet gravity}{
   April 2020}
%-----------------------------------------------------------------------------
\datedtalk{University of Illinois, Urbana-Champaign, IL}{
   Testing General Relativity and the nonlinear dynamics of modified
   gravity theories}{
   January 2020}
%-----------------------------------------------------------------------------
\datedtalk{Massachusetts Institute of Technology, Cambridge, MA}{
   Second order vacuum perturbation of a Kerr black hole}{
   December 2019}
%-----------------------------------------------------------------------------
\datedtalk{Black Hole Initiative, Harvard University, Cambridge, MA}{
   Nonlinear dynamics of Horndeski theories in spherical collapse}{
   December 2019}
\end{etaremune}
%-----------------------------------------------------------------------------
%-----------------------------------------------------------------------------
{\bf Contributed conference talks/seminars}
\begin{etaremune}
%-----------------------------------------------------------------------------
\datedtalk{GR23 (online)}{
   Evolution of binary scalar-hairy black holes}{
   July 2022}
%-----------------------------------------------------------------------------
\datedtalk{EPS-HEP2021 Conference (online)}{
   Modeling black hole binaries in scalar-tensor theories of gravity}{
   July 2021}
%-----------------------------------------------------------------------------
\datedtalk{APS April Meeting, Sacramento, CA (online)}{
   Application of the modified generalized harmonic formulation
   to scalar-tensor gravity theories}{
   April 2021}
%-----------------------------------------------------------------------------
\datedtalk{BritGrav21, UCD, Dublin, Ireland (online)}{
   Computing the second order vacuum perturbation of Kerr
   black holes}{
   April 2021}
%-----------------------------------------------------------------------------
\datedtalk{XIII Black Holes Workshop, IST, Lisbon, PT (online)}{
   Computing the second order vacuum perturbation of a Kerr
   black hole}{
   December 2020}
%-----------------------------------------------------------------------------
%\datedtalk{Midwest Relativity Meeting, University of Notre Dame (online)}{
%   Computing the second order gravitational perturbation of a Kerr
%   black hole}{
%   October 2020}
%-----------------------------------------------------------------------------
\datedtalk{APS April Meeting, Washington, DC (online)}{
   Second order perturbation of a Kerr black hole}{
   April 2020}
%-----------------------------------------------------------------------------
\datedtalk{GR 22/Amaldi 13, Valencia, Spain}{
   Nonlinear dynamics of Horndeski theories in spherical collapse}{
   July 2019}
%-----------------------------------------------------------------------------
\datedtalk{APS April Meeting, Denver, CO}{
   Hyperbolicity in gravitational collapse in a modified gravity theory}{
   April 2019}
%-----------------------------------------------------------------------------
\datedtalk{Numerical Relativity beyond General Relativity, Benasque, Spain}{
   Gravitational collapse in a modified gravity theory}{
   June 2018}
%-----------------------------------------------------------------------------
\end{etaremune}
%=============================================================================
\end{document}
